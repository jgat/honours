\mode<presentation>
{
\usetheme{Frankfurt}
\usecolortheme{beaver}

%\setbeamercovered{transparent}
}

\usepackage[english]{babel}
\usepackage[latin1]{inputenc}

% font definitions, try \usepackage{ae} instead of the following
% three lines if you don't like this look
\usepackage{mathptmx}
\usepackage[scaled=.90]{helvet}
\usepackage{courier}

\usepackage[T1]{fontenc}

\usepackage{amsmath}
\usepackage{amssymb}
\usepackage{thmtools}

\usepackage{tikz}
\usetikzlibrary[topaths]
\tikzstyle{every node}=[circle, draw, %fill=black!50,
                        inner sep=0pt, minimum width=12pt]
% A counter, since TikZ is not clever enough (yet) to handle
% arbitrary angle systems.
\newcount\mycount

\usepackage{wasysym}

\let\problem\relax
\declaretheorem[style=definition]{problem}
\theoremstyle{definition}
\newtheorem{question}[theorem]{\translate{Question}}

\newcommand{\floor}[1]{\left \lfloor #1 \right \rfloor}

\title{Decomposing Complete Hypergraphs}

%\subtitle{}

\author{Jackson Gatenby \\ \vspace{0.3cm} Supervisor: Barbara Maenhaut}

\date{17 October 2014}

% Delete this, if you do not want the table of contents to pop up at
% the beginning of each subsection:
%\AtBeginSubsection[]
%{
%\begin{frame}<beamer>
%\frametitle{Outline}
%\tableofcontents[currentsection,currentsubsection]
%\end{frame}
%}

% If you wish to uncover everything in a step-wise fashion, uncomment
% the following command:

%\beamerdefaultoverlayspecification{<+->}

\begin{document}

\begin{frame}
\titlepage
\end{frame}

\section{Steiner Triple Systems}
\subsection{}

\begin{frame}
\begin{problem}[Steiner, 1853]
Given a set of $v$ elements, take subsets of size $3$ so that every pair of
elements occurs in precisely one of the subsets.
\end{problem}

\pause

Example: every pair of elements from $\{0, 1, 2, 3, 4, 5, 6\}$ occurs in
exactly one of the following triples:
\begin{align*}
    \{0, 1, 3\} \quad &\{1, 2, 4\} \quad \{2, 3, 5\} \\
    \{3, 4, 6\} \quad &\{0, 4, 5\} \quad \{1, 5, 6\} \\
    &\{0, 2, 6\}
\end{align*}

\pause

\begin{definition}
A {\em Steiner Triple System} of order $v$, denoted $STS(v)$, is a
set $V$ of $v$ {\em points} with a collection $\mathcal{B}$ of 3-element
subsets of $V$, called {\em blocks}, so that every 2-element subset of $V$ is
contained in exactly one block.
\end{definition}

\end{frame}

\begin{frame}

%How can we visualise this?

\begin{center}
\begin{tikzpicture}[thick,scale=0.8]
  \foreach \number in {0,...,6}{
      % Compute angle:
        \mycount=\number
        \multiply\mycount by -360
        \divide\mycount by 7
        \advance\mycount by 90
      \node[draw,circle] (N-\number) at (\the\mycount:3cm) {\number};
    }

  \onslide*<2,9>{
      \draw (N-0) edge[-,red] (N-1);
      \draw (N-1) edge[-,blue] (N-3);
      \draw (N-3) edge[-,green] (N-0);
  }

  \onslide*<3,9>{
      \draw (N-1) edge[-,red] (N-2);
      \draw (N-2) edge[-,blue] (N-4);
      \draw (N-4) edge[-,green] (N-1);
  }

  \onslide*<4,9>{
      \draw (N-2) edge[-,red] (N-3);
      \draw (N-3) edge[-,blue] (N-5);
      \draw (N-5) edge[-,green] (N-2);
  }

  \onslide*<5,9>{
      \draw (N-3) edge[-,red] (N-4);
      \draw (N-4) edge[-,blue] (N-6);
      \draw (N-6) edge[-,green] (N-3);
  }

  \onslide*<6,9>{
      \draw (N-4) edge[-,red] (N-5);
      \draw (N-5) edge[-,blue] (N-0);
      \draw (N-0) edge[-,green] (N-4);
  }

  \onslide*<7,9>{
      \draw (N-5) edge[-,red] (N-6);
      \draw (N-6) edge[-,blue] (N-1);
      \draw (N-1) edge[-,green] (N-5);
  }

  \onslide*<8,9>{
      \draw (N-6) edge[-,red] (N-0);
      \draw (N-0) edge[-,blue] (N-2);
      \draw (N-2) edge[-,green] (N-6);
  }
\end{tikzpicture}
\end{center}

\onslide+<9>{$\Rightarrow$ We can draw the complete graph $K_7$ as
a number of copies of $K_3$.}

%\begin{problem}
%Construct a set of graphs isomorphic to $K_3$ whose edges partition the edge
%set of the complete graph $K_v$.
%\end{problem}

\end{frame}

\begin{frame}

\begin{definition}
Let $G, H$ be graphs. A $G$-decomposition of $H$ is a collection of graphs
isomorphic to $G$ whose edges partition $E(H)$.
\end{definition}

%\pause

\begin{lemma}
An $STS(v)$ is equivalent to a $K_3$-decomposition of $K_v$.
\end{lemma}

\end{frame}

\begin{frame}

An $STS(9)$, expressed as a $K_3$-decomposition of $K_{9}$:

\begin{center}
\begin{tikzpicture}[thick,scale=0.8]
  \foreach \number in {0,...,7}{
      % Compute angle:
        \mycount=\number
        \multiply\mycount by -360
        \divide\mycount by 8
        \advance\mycount by 90
      \node[draw,circle] (N-\number) at (\the\mycount:2cm) {\number};
    }
  \node[draw,circle] (infty) at (0:0cm) {$\infty$};

      \draw (N-0) edge[-,red] (N-1);
      \draw (N-1) edge[-,blue] (N-3);
      \draw (N-3) edge[-,green] (N-0);
\end{tikzpicture}
\quad \quad \quad
\begin{tikzpicture}[thick,scale=0.8]
  \foreach \number in {0,...,7}{
      % Compute angle:
        \mycount=\number
        \multiply\mycount by -360
        \divide\mycount by 8
        \advance\mycount by 90
      \node[draw,circle] (N-\number) at (\the\mycount:2cm) {\number};
    }
  \node[draw,circle] (infty) at (0:0cm) {$\infty$};

      \draw (N-0) edge[-,orange,bend left] (N-4);
      \draw (N-4) edge[-,violet] (infty);
      \draw (infty) edge[-,violet] (N-0);
\end{tikzpicture}
\end{center}

%\pause
$\{0, 1, 3\}$, $\{1, 2, 4\}$, $\{2, 3, 5\}$, \ldots
\hspace{1cm}% \pause
$\{0,4,\infty\}$, $\{1, 5, \infty\}$, \ldots

\end{frame}

\begin{frame}

We also have necessary conditions for the existence of a $K_3$-decomposition of $K_v$:

\begin{enumerate}
    \pause
    \item $K_v$ has ${v \choose 2}$ edges, which are partitioned
    into parts of size $|E(K_3)| = 3$, so $3$ must divide ${v \choose 2} = \frac{v(v-1)}{2}$.
    \pause
    \item Consider a vertex $x \in V(K_v)$. Whenever $x$ appears in a copy of
    $K_3$, there are 2 edges incident with $x$. There are a total of
    $\deg_{K_v}(x) = v-1$ edges incident with $x$, so $2$ must divide $v-1$.
\end{enumerate}
\pause

\begin{theorem}[Kirkman, 1847]
An $STS(v)$ exists iff $v \equiv 1$ or $3$ (mod $6$).
\end{theorem}

%\pause

\begin{proof}
($\Rightarrow$) Obvious necessary conditions above.

($\Leftarrow$) Generalise the constructions shown before.
\end{proof}

\end{frame}

\section{Steiner Systems}
\subsection{}

\begin{frame}

We can generalise ``$3$-element blocks'' to ``$k$-element blocks'', and
from ``covering every pair of points'' to ``covering every $t$-set of points'':

\pause

\begin{definition}
A {\em Steiner system}, denoted $S(t, k, v)$, consists of a set of $v$ points,
and sets of $k$ points ({\em blocks}), so that every set of $t$ points occurs
in precisely one block.
\end{definition}

\pause

\begin{problem}
Given $k$ and $t$, for which $v$ does there exist an $S(t, k, v)$?
\end{problem}

\pause
We would like to form an equivalent `graph theoretic' question.
%We need a `graph' which contains one edge for every {\em $t$-set}
%of points.

\end{frame}

\begin{frame}

%The following definition removes the restriction that edges of a graph need
%exactly two endpoints:
%\pause

\begin{definition}
A {\em hypergraph} is a pair $(V, E)$, where $V$ is a set of {\em vertices}
and $E$ is a set of nonempty subsets of $V$ called {\em hyperedges} (or
{\em edges}).

%\pause
A hypergraph is {\em $t$-uniform} if every edge has size $t$.

%\pause
The {\em complete $t$-uniform hypergraph of order $v$}, denoted $K_v^{(t)}$,
consists of $v$ vertices and edges given by every $t$-set of vertices.
\end{definition}

\pause

%For example, a $2$-uniform hypergraph is a graph, and $K_v = K_v^{(2)}$.

Example: $K_4^{(3)}$:

\pause

\begin{center}
\begin{tikzpicture}[thick,scale=0.8]
  \foreach \number in {0,...,3}{
      % Compute angle:
        \mycount=\number
        \multiply\mycount by -360
        \divide\mycount by 4
        \advance\mycount by 135
      \node[draw,circle] (N-\number) at (\the\mycount:2cm) {};
    }

  \draw (N-0) edge[-,red,bend left=8] (N-1);
  \draw (N-1) edge[-,red,bend left=8] (N-2);
  \draw (N-2) edge[-,red,bend left=8] (N-0);

  \pause
  \draw (N-0) edge[-,blue,bend right=8] (N-1);
  \draw (N-1) edge[-,blue,bend right=8] (N-3);
  \draw (N-3) edge[-,blue,bend right=8] (N-0);

  \pause
  \draw (N-0) edge[-,green,bend left=8] (N-2);
  \draw (N-2) edge[-,green,bend left=8] (N-3);
  \draw (N-3) edge[-,green,bend left=8] (N-0);

  \pause
  \draw (N-1) edge[-,brown,bend right=8] (N-2);
  \draw (N-2) edge[-,brown,bend right=8] (N-3);
  \draw (N-3) edge[-,brown,bend right=8] (N-1);
\end{tikzpicture}
\end{center}

\end{frame}


\begin{frame}

\begin{lemma}
An $S(t, k, v)$ is a $K_{k}^{(t)}$-decomposition of $K_{v}^{(t)}$.
\end{lemma}

\pause

The necessary divisibility conditions for the existence of an $S(t, k, v)$ are:

\begin{enumerate}
    \item $|E(K_k^{(t)})| = {k \choose t}$ divides $|E(K_v^{(t)})| = {v \choose t}$
    \pause
    \item For $x \in V(K_v^{(t)})$: $\deg_{K_k^{(t)}}(x) = {k-1 \choose t-1}$
    divides $\deg_{K_v^{(t)}}(x) = {v-1 \choose t-1}$
    \pause
    \item For $x, y$: $\deg_{K_k^{(t)}}(\{x, y\}) = {k-2 \choose t-2}$
    divides $\deg_{K_v^{(t)}}(\{x, y\}) = {v-2 \choose t-2}$
    \pause
    \item \ldots
\end{enumerate}

\pause
In summary:

For $i \geq 0$, ${k-i \choose t-i}$ must divide ${v-i \choose t-i}$.

\end{frame}

\section{Hypergraph Decompositions}
\subsection{}

\begin{frame}
\begin{problem}
Given a $t$-uniform hypergraph $H$, for which $v$ does there exist an
$H$-decomposition of $K_v^{(t)}$?
\end{problem}

These can also be referred to as {\em $H$-designs of order $v$}.

\pause

Example: for the hypergraph $H$ below, does there exist an $H$-decomposition of
$K_{6}^{(3)}$?

\begin{center}
\begin{tikzpicture}[thick,scale=0.8]
  \foreach \number in {0,...,5}{
      % Compute angle:
        \mycount=\number
        \multiply\mycount by -360
        \divide\mycount by 6
        \advance\mycount by 90
      \node[draw,circle] (N-\number) at (\the\mycount:2cm) {};
    }

  \draw (N-0) edge[-,red,bend right=8] (N-1);
  \draw (N-1) edge[-,red,bend right=8] (N-5);
  \draw (N-5) edge[-,red,bend right=8] (N-0);

  \draw (N-0) edge[-,blue,bend right=8] (N-2);
  \draw (N-2) edge[-,blue,bend right=8] (N-4);
  \draw (N-4) edge[-,blue,bend right=8] (N-0);

  \draw (N-1) edge[-,green,bend right=8] (N-3);
  \draw (N-3) edge[-,green,bend right=8] (N-5);
  \draw (N-5) edge[-,green,bend right=8] (N-1);

  \draw (N-2) edge[-,brown,bend right=8] (N-3);
  \draw (N-3) edge[-,brown,bend right=8] (N-4);
  \draw (N-4) edge[-,brown,bend right=8] (N-2);
\end{tikzpicture}
\end{center}

%\vspace{5cm}

%\pause
%\vspace{1cm}
%{\em TODO: draw an $H$-design of order 6 where $E(H) = \{\{1,2,3\},\{1,5,6\},\{2,3,4\},\{4,5,6\}\}$
%(the regular $3$-uniform hypergraph on 4 edges which I solved the existence problem for).

%During the talk, I would also draw a copy of $H$ on a whiteboard, so I can refer to it
%in later slides without drawing it in every single slide.}
% TODO Example: {}-design of order 6 - the one with the four edges.
\end{frame}

\begin{frame}

\begin{center}
\begin{tikzpicture}[thick,scale=0.8]
  \foreach \number in {0,...,4}{
      % Compute angle:
        \mycount=\number
        \multiply\mycount by -360
        \divide\mycount by 5
        \advance\mycount by 90
      \node[draw,circle] (N-\number) at (\the\mycount:3cm) {\number};
    }
  \node[draw,circle] (infty) at (0:0cm) {$\infty$};

  \onslide*<1>{
      \draw (N-4) edge[-,red,bend right=8] (N-0);
      \draw (N-0) edge[-,red,bend right=8] (N-1);
      \draw (N-1) edge[-,red,bend right=8] (N-4);

      \draw (N-0) edge[-,blue,bend left=8] (N-2);
      \draw (N-2) edge[-,blue,bend left=8] (N-3);
      \draw (N-3) edge[-,blue,bend left=8] (N-0);

      \draw (N-4) edge[-,green,bend right=8] (N-1);
      \draw (N-1) edge[-,green,bend left=8] (infty);
      \draw (infty) edge[-,green,bend left=8] (N-4);

      \draw (N-2) edge[-,brown,bend right=8] (N-3);
      \draw (N-3) edge[-,brown,bend right=8] (infty);
      \draw (infty) edge[-,brown,bend right=8] (N-2);
  }

  \onslide*<2>{
      \draw (N-0) edge[-,red,bend right=8] (N-1);
      \draw (N-1) edge[-,red,bend right=8] (N-2);
      \draw (N-2) edge[-,red,bend right=8] (N-0);

      \draw (N-1) edge[-,blue,bend left=8] (N-3);
      \draw (N-3) edge[-,blue,bend left=8] (N-4);
      \draw (N-4) edge[-,blue,bend left=8] (N-1);

      \draw (N-0) edge[-,green,bend right=8] (N-2);
      \draw (N-2) edge[-,green,bend left=8] (infty);
      \draw (infty) edge[-,green,bend left=8] (N-0);

      \draw (N-3) edge[-,brown,bend right=8] (N-4);
      \draw (N-4) edge[-,brown,bend right=8] (infty);
      \draw (infty) edge[-,brown,bend right=8] (N-3);
  }

  \onslide*<3>{
      \draw (N-1) edge[-,red,bend right=8] (N-2);
      \draw (N-2) edge[-,red,bend right=8] (N-3);
      \draw (N-3) edge[-,red,bend right=8] (N-1);

      \draw (N-2) edge[-,blue,bend left=8] (N-4);
      \draw (N-4) edge[-,blue,bend left=8] (N-0);
      \draw (N-0) edge[-,blue,bend left=8] (N-2);

      \draw (N-1) edge[-,green,bend right=8] (N-3);
      \draw (N-3) edge[-,green,bend left=8] (infty);
      \draw (infty) edge[-,green,bend left=8] (N-1);

      \draw (N-4) edge[-,brown,bend right=8] (N-0);
      \draw (N-0) edge[-,brown,bend right=8] (infty);
      \draw (infty) edge[-,brown,bend right=8] (N-4);
  }

  \onslide*<4>{
      \draw (N-2) edge[-,red,bend right=8] (N-3);
      \draw (N-3) edge[-,red,bend right=8] (N-4);
      \draw (N-4) edge[-,red,bend right=8] (N-2);

      \draw (N-3) edge[-,blue,bend left=8] (N-0);
      \draw (N-0) edge[-,blue,bend left=8] (N-1);
      \draw (N-1) edge[-,blue,bend left=8] (N-3);

      \draw (N-2) edge[-,green,bend right=8] (N-4);
      \draw (N-4) edge[-,green,bend left=8] (infty);
      \draw (infty) edge[-,green,bend left=8] (N-2);

      \draw (N-0) edge[-,brown,bend right=8] (N-1);
      \draw (N-1) edge[-,brown,bend right=8] (infty);
      \draw (infty) edge[-,brown,bend right=8] (N-0);
  }

  \onslide*<5>{
      \draw (N-3) edge[-,red,bend right=8] (N-4);
      \draw (N-4) edge[-,red,bend right=8] (N-0);
      \draw (N-0) edge[-,red,bend right=8] (N-3);

      \draw (N-4) edge[-,blue,bend left=8] (N-1);
      \draw (N-1) edge[-,blue,bend left=8] (N-2);
      \draw (N-2) edge[-,blue,bend left=8] (N-4);

      \draw (N-3) edge[-,green,bend right=8] (N-0);
      \draw (N-0) edge[-,green,bend left=8] (infty);
      \draw (infty) edge[-,green,bend left=8] (N-3);

      \draw (N-1) edge[-,brown,bend right=8] (N-2);
      \draw (N-2) edge[-,brown,bend right=8] (infty);
      \draw (infty) edge[-,brown,bend right=8] (N-1);
  }
\end{tikzpicture}
\end{center}

\end{frame}

\begin{frame}


%For $S \subseteq V(H)$, let $\deg_H(S) = |\{e \in E(H) : S \subseteq e\}|$.
%For example, $\deg_H(\{x\}) = \deg_H(x)$.

The necessary divisibility conditions for $H$-decompositions of $K_v^{(t)}$
are:
For all $i \geq 0$, $\gcd \left( \{\deg_H(S) : S \subseteq V(H), |S| = i\}
\right)$ must divide ${v-i \choose t-i}$.

\pause
\begin{example}
If $H$ is as before, and there is a $H$-design of order $v$, then
\begin{enumerate}
    \item $\deg_H(\emptyset) = |E(H)| = 4$ divides ${v \choose 3}$
    \pause
    \item For all $x \in V(H)$, $\deg_H(x) = 2$. So $2$ divides ${v-1 \choose 2}$
    \pause
    \item $\{\deg_H(\{x, y\}) : x, y \in V(H), x \neq y\} = \{1, 2\}$.
    So $\gcd(\{1, 2\}) = 1$ divides ${v-2 \choose 1} = v-2$.
\end{enumerate}
\pause
These conditions are equivalent to $v \equiv 1$, $2$ or $6$ (mod $8$).
\end{example}

Does there always exist an $H$-design for these values of $v$?

\end{frame}

%\section{Constructions}
%\subsection{}

\begin{frame}
If $v \equiv 1$ (mod $8$), then we need $n$ sets of 8 points, plus one extra
point:

\begin{center}
\tikzstyle{every node}=[]
\begin{tikzpicture}[thick,scale=0.8]
  \draw[semithick] (0,0) ellipse (1 and 2.5);
  \draw[semithick] (3,0) ellipse (1 and 2.5);
  \draw[semithick] (6,0) ellipse (1 and 2.5);
  \node at (8,0) {$\cdots$};
  \draw[semithick] (10,0) ellipse (1 and 2.5);
\tikzstyle{every node}=[circle, draw, fill=black!50, inner sep=0pt, minimum width=3pt]
  \node (infty) at (5,-4) {};

  \foreach \n in {0,3,6,10}{
    \node (\n-0) at (\n,1.75) {};
    \node (\n-1) at (\n,1.25) {};
    \node (\n-2) at (\n,0.75) {};
    \node (\n-3) at (\n,0.25) {};
    \node (\n-4) at (\n,-0.25) {};
    \node (\n-5) at (\n,-0.75) {};
    \node (\n-6) at (\n,-1.25) {};
    \node (\n-7) at (\n,-1.75) {};
  }

  \onslide*<2>{
    \draw (0-0) edge[-,blue] (3-2);
    \draw (3-2) edge[-,blue] (6-1);
    \draw (6-1) edge[-,blue] (0-0);

    \draw (3-4) edge[-,blue] (6-2);
    \draw (6-2) edge[-,blue] (10-2);
    \draw (10-2) edge[-,blue] (3-4);
  }

  \onslide*<3>{
    \draw (0-4) edge[-,orange] (3-3);
    \draw (3-3) edge[-,orange] (3-6);
    \draw (3-6) edge[-,orange] (0-4);

    \draw (3-0) edge[-,orange] (10-3);
    \draw (10-3) edge[-,orange] (3-1);
    \draw (3-1) edge[-,orange] (3-0);

    \draw (3-7) edge[-,orange] (6-5);
    \draw (6-5) edge[-,orange] (infty);
    \draw (infty) edge[-,orange] (3-7);
  }

  \onslide*<4>{
    \draw (6-0) edge[-,violet,bend left] (6-2);
    \draw (6-2) edge[-,violet,bend left] (6-5);
    \draw (6-5) edge[-,violet,bend left] (6-0);

    \draw (3-4) edge[-,violet] (3-6);
    \draw (3-6) edge[-,violet] (infty);
    \draw (infty) edge[-,violet] (3-4);
  }
\end{tikzpicture}
\end{center}

\end{frame}

\begin{frame}

If we can find $H$-decompositions of the following hypergraphs, then multiple
copies of them will cover all three cases:

\begin{center}
\tikzstyle{every node}=[circle, draw, fill=black!50, inner sep=0pt, minimum width=3pt]
\begin{tikzpicture}[thick,scale=0.8]
  \draw[semithick] (0,0) ellipse (0.5 and 2.25);
    \node (0) at (0,1.75) {};
    \node (1) at (0,1.25) {};
    \node (2) at (0,0.75) {};
    \node (3) at (0,0.25) {};
    \node (4) at (0,-0.25) {};
    \node (5) at (0,-0.75) {};
    \node (6) at (0,-1.25) {};
    \node (7) at (0,-1.75) {};
    \node (infty) at (0.75,-3) {};

    \draw (0) edge[-,violet,bend left] (2);
    \draw (2) edge[-,violet,bend left] (5);
    \draw (5) edge[-,violet,bend left] (0);

    \draw (6) edge[-,violet] (7);
    \draw (7) edge[-,violet] (infty);
    \draw (infty) edge[-,violet] (6);

    %\draw (3) edge[-,violet] (4);
    %\draw (4) edge[-,violet] (infty);
    %\draw (infty) edge[-,violet] (3);
\end{tikzpicture}
\quad \quad \quad \pause
\begin{tikzpicture}[thick,xscale=0.6,yscale=0.8]
  \draw[semithick] (0,0) ellipse (0.666667 and 2.25);
  \draw[semithick] (2,0) ellipse (0.666667 and 2.25);
  \foreach \n in {0,1}{
    \mycount=\n
    \multiply\mycount by 2
    \node (\n-0) at (\the\mycount,1.75) {};
    \node (\n-1) at (\the\mycount,1.25) {};
    \node (\n-2) at (\the\mycount,0.75) {};
    \node (\n-3) at (\the\mycount,0.25) {};
    \node (\n-4) at (\the\mycount,-0.25) {};
    \node (\n-5) at (\the\mycount,-0.75) {};
    \node (\n-6) at (\the\mycount,-1.25) {};
    \node (\n-7) at (\the\mycount,-1.75) {};
  }
    \node (infty) at (1,-3) {};

    \draw (0-4) edge[-,orange] (1-3);
    \draw (1-3) edge[-,orange] (1-5);
    \draw (1-5) edge[-,orange] (0-4);

    \draw (0-0) edge[-,orange] (1-2);
    \draw (1-2) edge[-,orange] (0-1);
    \draw (0-1) edge[-,orange] (0-0);

    \draw (0-7) edge[-,orange] (1-6);
    \draw (1-6) edge[-,orange] (infty);
    \draw (infty) edge[-,orange] (0-7);
\end{tikzpicture}
\quad \quad \quad \pause
\begin{tikzpicture}[thick,xscale=0.6,yscale=0.8]
  \draw[semithick] (0,0) ellipse (0.666667 and 2.25);
  \draw[semithick] (2,0) ellipse (0.666667 and 2.25);
  \draw[semithick] (4,0) ellipse (0.666667 and 2.25);
  \foreach \n in {0,1,2}{
    \mycount=\n
    \multiply\mycount by 2
    \node (\n-0) at (\the\mycount,1.75) {};
    \node (\n-1) at (\the\mycount,1.25) {};
    \node (\n-2) at (\the\mycount,0.75) {};
    \node (\n-3) at (\the\mycount,0.25) {};
    \node (\n-4) at (\the\mycount,-0.25) {};
    \node (\n-5) at (\the\mycount,-0.75) {};
    \node (\n-6) at (\the\mycount,-1.25) {};
    \node (\n-7) at (\the\mycount,-1.75) {};
  }
    \tikzstyle{every node}=[]
    \node (infty) at (2,-3) {};

    \draw (0-0) edge[-,blue] (1-2);
    \draw (1-2) edge[-,blue] (2-1);
    \draw (2-1) edge[-,blue] (0-0);

    \draw (1-4) edge[-,blue] (2-5);
    \draw (2-5) edge[-,blue] (0-5);
    \draw (0-5) edge[-,blue] (1-4);
\end{tikzpicture}
\end{center}

\end{frame}

\begin{frame}

It can be shown that all of these exist, so $H$-designs of order $v$ exist for
all $v \equiv 1$ (mod $8$). Similarly for $v \equiv 2$ and $6$, so

\pause

\begin{theorem}
For the hypergraph $H$ given before, there is an $H$-design of order $v$
($H$-decomposition of $K_v^{(3)}$) iff $v \equiv 1$, $2$, or $6$ (mod $8$).
\end{theorem}

\pause

\begin{proof}
$(\Rightarrow)$ Necessary divisibility conditions.

$(\Leftarrow)$ Divide into three cases: $1$, $2$, $6$ (mod $8$).

Find $H$-decompositions of certain small hypergraphs.

Combine copies of these decompositions as necessary.
\end{proof}

%Similarly, the existence of $H$-designs has been solved for:
%\begin{enumerate}
%    \item All 3-uniform hypergraphs on at most 3 edges.
%    \item The two regular 3-uniform hypergraphs on 4 edges.
%\end{enumerate}

\end{frame}

\end{document}
