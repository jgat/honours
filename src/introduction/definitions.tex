\section{Definitions}

% TODO introductory sentence/paragraph?

A {\em graph} $G$ is a pair $(V, E)$, where $V = V(H)$ is a finite set of {\em vertices}, and $E = E(H)$ is a set of $2$-element subsets of $V$, called {\em edges}.
A {\em subgraph} of $G$ is a graph $H$ where $V(H) \subseteq V(G)$ and $E(H) \subseteq E(G)$.

Two graphs $G$ and $H$ are {\em isomorphic} if there is a bijection $\varphi : V(G) \to V(H)$ such that for every edge $\{u, v\} \in E(G)$, $\{\varphi(u), \varphi(v)\} \in E(H)$, and for every edge $\{x, y\} \in E(H)$, $\{\varphi^{-1}(x), \varphi^{-1}(y)\} \in E(G)$. The map $\varphi$ is called an {\em isomorphism} from $G$ to $H$.

%(Note that in order for $H$ to be a well-defined graph, we must also have $e \subseteq V(H)$ for each edge $e \in E(H)$.)

A {\em multigraph} $G$ is a pair $(V, E)$ where $V = V(G)$ is a finite set of vertices, and $E = E(G)$ is a finite multiset of $2$-element subsets of $V$, called edges. That is, we allow for edges to occur multiple times in the
edge set, but only finitely many copies of each edge.

A {\em decomposition} of a graph or multigraph $G$ is a set $\mathcal{D} = \{G_1, G_2, \ldots, G_d\}$ of
subgraphs of $G$ such that $\bigcup_{i=1}^d E(G_i) = E(G)$ (hence, if $G$ is not a multigraph, then the $G_i$ are pairwise edge-disjoint).
If $\Gamma$ is a family of graphs such that each $G_i \in \mathcal{D}$ is isomorphic to some
member of $\Gamma$, then $\mathcal{D}$ is said to be a $\Gamma$-decomposition of $G$.
If $\Gamma$ contains one element $H$, then $\mathcal{D}$ is said to be an
{\em $H$-decomposition} of $G$.

The reader is referred to \cite{bryant-graph} for a discussion on known results
in graph decompositions.

If we observe that each edge of a graph is merely a 2-element subset of the
vertex set, it leads to a natural generalisation.
A {\em hypergraph} $H$ is a pair $(V, E)$, where $V = V(H)$ is a finite set of
{\em vertices}, and $E = E(H)$ is a set of subsets of $V$, called {\em
hyperedges} or {\em edges}.
The {\em order} of $H$ is $|V|$, and the {\em size} of $H$ is $|E|$.
$H$ is said to be {\em $t$-uniform} if each edge has size $t$.

Let $S \subseteq V(H)$ be a set of vertices in $H$. The {\em degree} of $S$,
denoted $\deg_H(S)$, is $|\{e \in E(H) : S \subseteq e\}|$, that is, the number
of edges which contain $S$.
For any vertex $x \in V(H)$, the degree of $x$ is $\deg_H(x) = \deg_H(\{x\})$.
For example, a $2$-uniform hypergraph is a graph, and in this case the notions
of the degree of a vertex $x$ coincide. Note that $\deg_H(\emptyset) = |E(H)|$.

A {\em subhypergraph} of $H$ is a hypergraph $K$ where $V(K) \subseteq V(H)$ and $E(K) \subseteq E(H)$. The subhypergraph $K$ is {\em spanning} if $V(K) = V(H)$.

Two hypergraphs $H$ and $K$ are {\em isomorphic} if there is a bijection
$\varphi : V(H) \to V(K)$ such that for every edge $e \in E(H)$,
$\varphi(e) = \{\varphi(x) : x \in e\} \in E(K)$, and for every
$f \in E(K)$, $\varphi^{-1}(f) \in E(H)$. The map $\varphi$ is called an {\em isomorphism} from $H$ to $K$.

The {\em union} of hypergraphs $H$ and $K$, denoted
$H \cup K$, is the hypergraph with $V(H \cup K) = V(H) \cup V(K)$ and
$E(H \cup K) = E(H) \cup E(K)$.

If $K$ is a subhypergraph of $H$, then their {\em difference}
is the hypergraph $H \setminus K$ with $V(H \setminus K) = V(H)$ and
$E(H \setminus K) = E(H) \setminus E(K)$.

A {\em multihypergraph} $H$ is a pair $(V, E)$ where $V = V(H)$ is a finite set
of vertices, and $E = E(H)$ is a finite multiset of subsets of $V$, called
hyperedges or edges.

A {\em decomposition} of a hypergraph or multihypergraph $K$ is a set $\mathcal{D} = \{K_1,
K_2, \ldots, K_d\}$ of subhypergraphs of $K$ such that
$\bigcup_{i=1}^d E(K_i) = E(K)$, where this union respects multiplicity (hence, if $K$ is not a multihypergraph, then the $K_i$ are pairwise edge-disjoint).
If each $K_i \in \mathcal{D}$ is isomorphic to some member
of a family $\Gamma$ of hypergraphs, then the decomposition is a {\em
$\Gamma$-decomposition} of $K$. If $\Gamma$ contains one member $H$, then the
decomposition is an {\em $H$-decomposition} of $K$.

\subsection{Examples of Hypergraphs}

In this section, we will describe some useful families of hypergraphs.

For a finite non-empty set $V$, the {\em complete $t$-uniform hypergraph on $V$}
is the hypergraph $K_V^{(t)} = (V, E)$ with $E = \{e \in \mathcal{P}(V) : |e| = k\}$.
For a positive integer $v$, the notation $K_v^{(t)}$ refers to any complete
$t$-uniform hypergraph of order $v$.
For example, we identify $K_v^{(2)}$ with the complete graph $K_v$.

For a finite non-empty set $V$ and a natural number $\lambda$, the {\em complete
$t$-uniform multihypergraph}, denoted $\lambda K_{V}^{(t)}$, is the
multihypergraph on $V$ containing $\lambda$ copies of every $t$-element subset
of $V$. For positive $v \in \Z$, the notation $\lambda K_{v}^{(t)}$ refers to
any complete $t$-uniform multihypergraph $\lambda K_{V}^{(t)}$ where $|V| = v$.

Given a collection of $m$ finite, non-empty, pairwise disjoint sets
$V_1, V_2, \ldots, V_m$, let $V = \bigcup_{i=1}^{m} V_i$.
The {\em complete $t$-uniform $m$-partite hypergraph} on $V$, denoted
$K_{V_1, V_2, \ldots, V_m}^{(t)}$, is the $t$-uniform hypergraph $(V, E)$ having
all edges which contain at most one element from each of $V_1, V_2, \ldots, V_m$.
Formally, $e \in E$ iff $|e| = t$ and for all $i=1,\ldots,m$, $|e \cap V_i| \leq 1$.
In such a hypergraph, the sets $V_1, V_2, \ldots, V_m$ are called the
{\em partite sets} of the hypergraph.

For positive integers $v_1, v_2, \ldots, v_m$, the notation $K_{v_1, v_2,
\ldots, v_m}^{(t)}$ is used to denote any complete $t$-uniform $m$-partite
hypergraph where the partite sets have order $v_1, v_2, \ldots, v_m$
respectively.
%If $v_1 = v_2 = \cdots = v_m = v$, then we may write $K_{m[v]}^{(t)}$ to denote $K_{v, v, \ldots, v}^{(t)}$.

For example, we identify $K_{v_1, v_2}^{(2)}$ with the complete bipartite graph
$K_{v_1,v_2}$.

Given a collection of $m+1$ finite pairwise disjoint sets
$V_1, V_2, \ldots, V_m, W$, where $V_i \neq \emptyset$ for each $i = 1, 2, \ldots, m$, we define
\[
    L_{V_1, V_2, \ldots, V_m, [W]}^{(t)} = K_{V_1 \cup V_2 \cup \cdots \cup V_m \cup W}^{(t)} \setminus (K_{V_1 \cup W}^{(t)} \cup K_{V_2 \cup W}^{(t)} \cup \cdots \cup K_{V_m \cup W}^{(t)}).
\]
That is, $L_{V_1, V_2, \ldots, V_m, [W]}^{(t)}$ is the $t$-uniform hypergraph with vertex set $V = V_1 \cup V_2 \cup \cdots \cup V_m \cup W$, and where $e \subseteq V$ is an edge if and only if $|e| = t$ and $e$ has non-empty intersection with at least two of $V_1, \ldots, V_m$.

For integers $v_1, v_2, \ldots, v_m \geq 1$ and $w \geq 0$, the notation $L_{v_1, v_2, \ldots, v_m, [w]}^{(t)}$ is used to denote any hypergraph $L_{V_1, V_2, \ldots, V_m, [W]}^{(t)}$ where $|V_i| = v_i$ for each $i = 1, 2, \ldots, m$, and $|W| = w$.

If $W = \emptyset$, we may use the notation $L_{V_1, V_2, \ldots, V_m}^{(t)} = L_{V_1, V_2, \ldots, V_m, [W]}^{(t)}$, and if $w = 0$, we may use the notation $L_{v_1, v_2, \ldots, v_m}^{(t)} = L_{v_1, v_2, \ldots, v_m, [0]}^{(t)}$.

%If $v_1 = v_2 = \cdots = v_m = v$, then we may write $L_{m[v]}^{(t)}$ to denote $L_{v, v, \ldots, v}^{(t)}$.

Given a $t$-uniform hypergraph $H$, a {\em $d$-factor} of $H$ is a spanning
subhypergraph $K$ where each vertex of $K$ has degree $d$. For example, a
1-factor consists of $\frac{|V(H)|}{t}$ pairwise disjoint edges from $H$. An
obvious necessary condition then for the existence of a 1-factor of $H$ is that
$t$ divides $|V(H)|$.
