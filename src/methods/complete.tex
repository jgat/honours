\section{Decompositions of $K_{v}^{(3)}$}

Let $H$ be a $3$-uniform hypergraph, and let $v \geq 3$ satisfy the conditions of Lemma \ref{necessary-3-uniform}. We would like to construct a $H$-decomposition $\mathcal{D}$ of $K_{v}^{(3)}$ (if one exists), with the condition that there is a set $X$ of $H$-blocks, and a non-trivial permutation $\pi$ acting on $V(K_{v}^{(3)})$ such that
\begin{equation} \label{eq:complete-decomposition-orbit}
    \mathcal{D} = \bigcup_{H' \in X} \mathcal{O}_\pi(H'),
\end{equation}
where $\mathcal{O}_{\pi}(H')$ denotes the orbit of $H'$ under $\pi$.
Moreover, we shall require that $\pi$ is a cyclic permutation (possibly containing fixed points), and that $\pi$ acts freely on $X$ (that is, all stabilisers are trivial).

We shall call $X$ a set of {\em base blocks} of $\mathcal{D}$ for $\mathcal{\pi}$.

If such a decomposition $\mathcal{D}$ exists, the following criteria must be satisfied:

\begin{lemma}
Suppose $\mathcal{D}$ is an $H$-decomposition of $K_{v}^{(3)}$ satisfying the conditions above, and let $|\pi|$ denote the order of the permutation $\pi$.
Then, ${v \choose 3} = |E(H)| \cdot |X| \cdot |\pi|$, and $v-2 \leq |\pi| \leq v$ (that is, $\pi$ has at most two fixed points).
\end{lemma}

\begin{proof} % TODO
Since $\pi$ acts freely on $X$, the total number of $H$-blocks in $\mathcal{D}$ is $|X| \cdot |\pi|$. The edge sets of these $H$-blocks partition $E(K_{v}^{(3)})$, so therefore
\[
    |E(K_{v}^{(3)})| = {v \choose 3} = |E(H)| \cdot |X| \cdot |\pi|.
\]

Since $\pi$ acts on $|V(K_{v}^{(3)})| = v$ elements, $|\pi| \leq v$.
To show that $|\pi| \geq v-2$, it suffices to show that $\pi$ has at most two fixed points.

Suppose for contradiction that $\pi$ has at least three fixed points, and let $\infty_1, \infty_2, \infty_3$ be fixed points of $\pi$. Let $H_0$ be the unique $H$-block of $\mathcal{D}$ which contains the edge $e = \{\infty_1, \infty_2, \infty_3\}$.
Since $\pi(e) = e$, $e$ is an edge of $\pi(H_0)$. But $H_0$ is unique, so $\pi(H_0) = H_0$, and since $\pi$ acts freely on $X$, $\pi$ is the identity permutation.
Since $\pi$ was assumed to be non-trivial, we have a contradiction, so $\pi$ has at most two fixed points.
\end{proof}

This gives us a method to find suitable permutations $\pi$, and to know in advance the value of $|X|$. Given $H$ and $v$, let $n = |\pi| \in \{v-2, v-1, v\}$ such that $n$ divides ${v \choose 3}/|E(H)|$. WLOG, we can label the vertices of $K_{v}^{(3)}$ so that
\begin{align} \label{eq:complete-vertices}
    V(K_{v}^{(3)}) &= \{0, 1, 2, \ldots, n-1\} \cup \{\infty_1, \ldots, \infty_{v-n}\}, \\
    \label{eq:complete-permutation}
    \pi &= (0\;1\;2\;\cdots\;n-1)(\infty_1)\cdots(\infty_{v-n}).
\end{align}
Equivalently, we identify $\pi$ with the action of $\Z_n$, including at most two fixed points, denoted $\infty_1$ and $\infty_2$.
Then, $|X| = {v \choose 3}/(|E(H)| \cdot n)$, and it only remains to find a suitable set $X$ of base blocks.

\begin{definition} \label{def:admissible-orbits}
Let $H$ be a $3$-uniform hypergraph, $v \geq 3$, and $\pi$ acting on $V(K_{v}^{(3)})$ be given by \eqref{eq:complete-permutation}.
Consider the orbit of each edge of $K_{v}^{(3)}$ under $\pi$, of which there are ${v \choose 3}/n$ distinct orbits.

For a set $B$ containing $|E(H)|$ of these orbits, say that $B$ is {\em admissible} if there exists a hypergraph $H' \simeq H$ such that $E(H')$ is a system of distinct representatives of $B$. Equivalently, $B$ is admissible if there exists a hypergraph $H'$ isomorphic to $H$ such that $B = \{\mathcal{O}_\pi(e) \mid e \in E(H')\}$.

If $B$ is an admissible set of edge orbits, then any hypergraph $H'$ satisfying the above conditions is said to be a hypergraph {\em corresponding to} $B$. Note that $H'$ is not necessarily unique.
\qed
\end{definition}

Hence, if we can partition $E(K_{v}^{(3)})$ into parts of size $|E(H)|$ such that every part is admissible, then such a partition will give rise to a set $X$ such that $\mathcal{D}$ given by \eqref{eq:complete-decomposition-orbit} is an $H$-decomposition of $K_{v}^{(3)}$. In particular, $X$ can be given by a set of hypergraphs corresponding to each part in the partition.

It then remains to describe the orbits of edges in $K_{v}^{(3)}$; we shall introduce the following definitions:

\begin{definition} \label{def:diff-triple}
Given a positive integer $n \geq 3$, let $A$ be the set of ordered triples $(a, b, c)$, where $1 \leq a, b, c \leq \floor{\frac{n}{2}}$, that are solutions to either
\[
    a + b \equiv c \quad \text{or} \quad a + b + c \equiv 0 \mod n.
\]
Define an equivalence relation $\sim$ on $A$ where $(a, b, c) \sim (x, y, z)$ if and only if $(a, b, c) = (x, y, z)$, or $(a, b, c) = (y, z, x)$, or $(a, b, c) = (z, x, y)$. That is, $\sim$ is equivalence up to a cyclic rearrangement of $(a, b, c)$.

The set of {\em difference triples} modulo $n$ is defined to be the set of equivalence classes $A/\sim$. We shall use the notation $(a, b, c)$ to refer to the equivalence class containing $(a, b, c)$.
\qed
\end{definition}

\begin{definition} \label{def:orbit-types}
Let $\pi$ act on $K_{v}^{(3)}$, given by \eqref{eq:complete-permutation}, and consider the orbit of each edge of $K_{v}^{(3)}$ under $\pi$. We define {\em types} of orbits as follows:

\begin{itemize}
    \item The orbit containing all edges of the form $\{\infty_1, \infty_2, i\}$ for $i \in \Z_n$ is of type $(\infty_1, \infty_2)$;
    \item For each $1 \leq d \leq \floor{\frac{n}{2}}$ and $a \in \{1, 2\}$, the orbit containing all edges of the form $\{\infty_a, i, i+d\}$ for $i \in \Z_n$ is of type $(\infty_a, d)$;
    \item For each difference triple $(a, b, c)$ modulo $n$, the orbit containing all edges of the form $\{i, i+a, i+a+b\}$ for $i \in \Z_n$ is of type $(a, b, c)$.
\end{itemize}

This classifies all orbits of edges $K_{v}^{(3)}$.
\qed
\end{definition}

% TODO: proof?

In order to generate difference triples modulo $n$, we use the following lemma:

\begin{lemma} \label{lem:diff-triple-existence}
Let $(a, b, c)$ be a difference triple modulo $n$.
It holds that $a + b \equiv c$ if and only if $a + b \leq \floor{\frac{n}{2}}$, and $a + b + c \equiv 0$ (mod $n$) if and only if $a + b \geq \ceil{\frac{n}{2}}$.

Moreover, $(b, c, a)$ and $(c, a, b)$ are difference triples equal to $(a, b, c)$ if and only if $a + b + c \equiv 0$ (mod $n$).
\end{lemma}

\begin{proof}
Recall that $0 \leq a, b, c \leq \floor{\frac{n}{2}}$.

If $a + b \equiv c$ (mod $n$), then $a + b = c \leq \floor{\frac{n}{2}}$.
If $a + b + c \equiv 0$ (mod $n$), then $a + b = n-c \geq n-\floor{\frac{n}{2}} = \ceil{\frac{n}{2}}$.
The converse is then immediate, except in the case that $n$ is even and $a + b = \frac{n}{2}$. But in this case, we also have $a + b + \frac{n}{2} = n \equiv 0$ (mod $n$), so we must have $c = \frac{n}{2}$.

Next, if $a + b + c \equiv 0$ (mod $n$), then clearly $(b, c, a)$ and $(c, a, b)$ are difference triples equal to $(a, b, c)$. Conversely, if $a + b + c \not\equiv 0$ (mod $n$), then $a + b = c < \frac{n}{2}$, so $2 \leq 2a, 2b < n$, so $b + c = a + 2b \not\equiv a$ and $c + a = 2a + b \not\equiv b$, therefore $(b, c, a)$ and $(c, a, b)$ are not difference triples.
\end{proof}

Given $H$ and $v$, we can now describe a method of generating $H$-decompositions of $K_{v}^{(3)}$ of the form \eqref{eq:complete-decomposition-orbit}:

First, given $v$ and $n = |\pi| \in \{v-2, v-1, v\}$, Algorithm \ref{alg:get-orbits} below will compute the set of all orbit types of $\pi$ acting on $K_{v}^{(3)}$, as given in Definition \ref{def:orbit-types}.
As a result of Lemma \ref{lem:diff-triple-existence}, we see that we have indeed generated all difference triples.

\begin{algorithm}
\KwIn{$v \geq 3$, $n \in \{v-2, v-1, v\}$}
\KwOut{The set of all orbit representations for a cyclic permutation of length $n$ on $K_v^{(3)}$, as described above.}

$\mathcal{O} \gets \emptyset$\;

\For{$a = 1$ \KwTo $\floor{\frac{n}{2}}$}{
    \For{$b = 1$ \KwTo $\floor{\frac{n}{2}}$}{
%        \tcc{Two cases: $a + b = c$ and $a + b + c = n$.}
        \uIf{$a + b < \ceil{\frac{n}{2}}$}{
            \tcc{$a + b + c \not\equiv 0$ (mod $n$)}
            $c \gets a + b$\;
            $\mathcal{O} \gets \mathcal{O} \cup \{(a, b, c)\}$\;
        }\uElse{
            $c \gets n - (a + b)$\;
            \tcc{We have $(a, b, c) = (b, c, a) = (c, a, b)$. WLOG, take the lexicographically smallest one.}
            \If{$(a, b, c) \leq (b, c, a) \land (a, b, c) \leq (c, a, b)$}{
                $\mathcal{O} \gets \mathcal{O} \cup \{(a, b, c)\}$
            }
        }
    }
}

\If{$n \leq v-1$}{
    \For{$d = 1$ \KwTo $\floor{\frac{n}{2}}$}{
        \For{$a = 1$ \KwTo $v-n$}{
            $\mathcal{O} \gets \mathcal{O} \cup \{(\infty_a, d)\}$\;
        }
    }
}

\If{$n = v-2$}{
    $\mathcal{O} \gets \mathcal{O} \cup \{(\infty_1, \infty_2)\}$\;
}

\KwRet{$\mathcal{O}$}

\caption{Find orbit types of $K_v^{(3)}$ under a cyclic permutation.} \label{alg:get-orbits}
\end{algorithm}

Next, recall that in order to generate the set of base blocks $X$, we require a mapping from admissible sets of edge orbits $B$ to hypergraphs $H'$ which correspond to $B$. Algorithm \ref{alg:get-possible-blocks} below constructs such a mapping.

\begin{algorithm}
\KwIn{$3$-uniform hypergraph $H$; $v \geq 3$; $n \in \{v-2, v-1, v\}$}
\KwOut{A mapping $c$ from admissible sets of orbits to one of the corresponding hypergraphs}

$c \gets$ empty mapping\;
$V_{all} = \{0, 1, \ldots, n-1\} \cup \{\infty_{1}, \ldots \infty_{v-n}\}$\;
\For{{\rm each ordered selection} $V$ {\rm of} $|V(H)|$ {\rm elements from} $V_{all}$}{
    \tcc{}
    $H' \gets$ {\tt copy}($H$, $V$) \tcc*{Isomorphic copy of $H$ on $V$}
    $B \gets \emptyset$\;
    \For{$e \in E(H')$} {
        \uIf{$e = \{u, v, w\}$: $u, v, w \in \Z_n$} {
            $t \gets$ $(a, b, c)$ such that $e=\{x,x+a,x+a+b\}$ for some $x$\;
            $B \gets B \cup \{t\}$\;
        } \uElseIf{$e = \{\infty_a, v, w\}$: $a \in \{1,2\}$, $v, w \in \Z_n$, $v < w$} {
            $d \gets \min \{w-v, n-(w-v)\}$\;
            $B \gets B \cup \{(\infty_a, d)\}$ \;
        } \uElseIf{$e = \{\infty_1, \infty_2, w\}$: $w \in \Z_n$} {
            $B \gets B \cup \{(\infty_1, \infty_2)\}$\;
        }
    }
    \If{$|B| = |E(H)|$ {\rm and} $c(B)$ {\rm is undefined}} {
        Let $c(B) = H'$\;
    }
}
\KwRet{$c$}

% TODO caption
\caption{Finding a mapping between admissible sets of orbits and corresponding $H$-blocks} \label{alg:get-possible-blocks}
\end{algorithm}

Finally, Algorithm \ref{alg:decompose-complete} describes a method for finding an $H$-decomposition of $K_{v}^{(3)}$, satisfying \eqref{eq:complete-decomposition-orbit}, if one exists.

\begin{algorithm}
\KwIn{$3$-uniform hypergraph $H$; $v \geq 3$}
\KwOut{The order $n$ of the cyclic permutation $\pi$, and a set $X$ of base blocks}
\SetKwFunction{getorbits}{get\_orbits}
\SetKwFunction{getadmissible}{get\_admissible}
\SetKwFunction{findpartition}{find\_partition}
\SetKwProg{procedure}{Procedure}{}{}

$n \gets \max(\{n \in \{v-2, v-1, v\} : n \mid {v \choose 3}/|E(H)|\})$\;
$\mathcal{O} \gets$ \getorbits{$v$, $n$} \tcc*{Algorithm \ref{alg:get-orbits}}
$c \gets$ \getadmissible{$H$, $v$, $n$} \tcc*{Algorithm \ref{alg:get-possible-blocks}}
$X \gets$ \findpartition{$\mathcal{O}$, $c$, $|E(H)|$} \tcc*{See below}
\KwRet{$n$, $X$}

\procedure{\findpartition{$\mathcal{O}$, $c$, $e$}}{
  \tcc{This recursive procedure will attempt to find a partition of $\mathcal{O}$ into admissible parts}
  \If{$\mathcal{O} = \emptyset$}{
    \KwRet{$\emptyset$}
  }
  \For{{\rm each} $B \subseteq \mathcal{O}$, $|B| = e$}{
    \If{$c(B)$ {\rm is defined}}{
        $X \gets$ \findpartition{$\mathcal{O} \setminus B$, $c$, $e$} \;
        \If{$X \neq$ ``No Solution''}{
            $X \gets X \cup \{c(B)\}$ \;
            \KwRet{$X$}
        }
    }
  }
  \KwRet{``No Solution''}
}

\caption{Finding $H$-decompositions of $K_{v}^{(3)}$} \label{alg:decompose-complete}

\end{algorithm}

These three algorithms have been implemented in Appendix \ref{appendix:code}, and were used to construct Examples \ref{eg:H_3i-9}, \ref{eg:H_3i-10}, and \ref{eg:H_3i-11}.
