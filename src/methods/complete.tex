\section{Decompositions of $K_{v}^{(3)}$}

Let $H$ be a $3$-uniform hypergraph, and let $v \geq 3$ be an admissible parameter for the existence of an $H$-design of order $v$.
We would like to fix a permutation $\pi$, and then aim to find an $H$-decomposition $\mathcal{D}$ of $K_{v}^{(3)}$ given by \eqref{eq:decomposition-orbit} (if one exists).

We shall choose $\pi$ to be a cyclic permutation (possibly containing fixed points). The following result restricts $\pi$ to having at most two fixed points:

\begin{lemma}
Suppose $\mathcal{D}$ is an $H$-decomposition of $K_{v}^{(3)}$ given by \eqref{eq:decomposition-orbit}, where $\pi$ is a non-trivial cyclic permutation. Then $v-2 \leq |\pi| \leq v$.
\end{lemma}

\begin{proof}
Since $\pi$ acts on $|V(K_{v}^{(3)})| = v$ elements, $|\pi| \leq v$.
To show that $|\pi| \geq v-2$, it suffices to show that $\pi$ has at most two fixed points.

Suppose for contradiction that $\pi$ has at least three fixed points, and let $\infty_1, \infty_2, \infty_3$ be distinct fixed points of $\pi$. Let $H_0$ be the unique $H$-block of $\mathcal{D}$ which contains the edge $e = \{\infty_1, \infty_2, \infty_3\}$.
Since $\pi(e) = e$, $e$ is an edge of $\pi(H_0)$. But $H_0$ is unique, so $\pi(H_0) = H_0$, and since $\pi$ acts freely on $X$, $\pi$ is the identity permutation.
Since $\pi$ was assumed to be non-trivial, we have a contradiction, so $\pi$ has at most two fixed points.
\end{proof}

Given $H$ and $v$, let $n = |\pi| \in \{v-2, v-1, v\}$ such that $n$ divides ${v \choose 3}/|E(H)|$. WLOG, we can label the vertices of $K_{v}^{(3)}$ so that
\begin{equation} \label{eq:complete-vertices}
    V(K_{v}^{(3)}) = \{0, 1, 2, \ldots, n-1\} \cup I,
\end{equation}
where $I = \emptyset$ if $n = v$, $I = \{\infty_1\}$ if $n = v-1$, and $I = \{\infty_1, \infty_2\}$ if $n = v-2$. Then, let
\begin{equation} \label{eq:complete-permutation}
    \pi =
    \begin{cases}
        (0\; 1\; 2\; \cdots\; n-1) & \text{if } n = v \\
        (0\; 1\; 2\; \cdots\; n-1) (\infty_1) & \text{if } n = v-1 \\
        (0\; 1\; 2\; \cdots\; n-1) (\infty_1) (\infty_2) & \text{if } n = v-2 \\
    \end{cases}
\end{equation}
So it only remains to find a suitable set $X$ of base blocks, where $|X| = {v \choose 3}/(|E(H)| \cdot n)$.

\begin{definition} \label{def:admissible-orbits}
Let $H$ and $K$ be $3$-uniform hypergraphs, and let $\pi$ be an automorphism of $K$.
Consider the orbit of each edge of $K$ under $\pi$.

For a set $B$ containing $|E(H)|$ of these orbits, say that $B$ is {\em admissible} if there exists a subhypergraph $H'$ of $K$ such that $H' \simeq H$ and $E(H')$ is a system of distinct representatives of $B$. Equivalently, $B$ is admissible if there exists a hypergraph $H'$ isomorphic to $H$ such that $B = \{\mathcal{O}_\pi(e) \mid e \in E(H')\}$.

If $B$ is an admissible set of edge orbits, then any hypergraph $H'$ satisfying the above conditions is said to be a hypergraph {\em corresponding to} $B$. Note that $H'$ is not necessarily unique.
\qed
\end{definition}

Hence, if we can partition $E(K_{v}^{(3)})$ into parts of size $|E(H)|$ such that every part is admissible, then such a partition will give rise to a set $X$ such that $\mathcal{D}$ given by \eqref{eq:decomposition-orbit} is an $H$-decomposition of $K_{v}^{(3)}$. In particular, $X$ can be described as a set of hypergraphs corresponding to each part in the partition.

It then remains to describe the orbits of edges in $K_{v}^{(3)}$; we shall introduce the following definitions:

\begin{definition} \label{def:diff-triple}
Given a positive integer $n \geq 3$, let $A$ be the set of ordered triples $(a, b, c)$, where $1 \leq a, b, c \leq \floor{\frac{n}{2}}$, that are solutions to either
\[
    a + b = c \quad \text{or} \quad a + b + c = n.
\]
(Note that these conditions will both hold if $n$ is even and $c = \frac{n}{2}$.)

Define an equivalence relation $\sim$ on $A$ where $(a, b, c) \sim (x, y, z)$ if and only if $(a, b, c) = (x, y, z)$, or $(a, b, c) = (y, z, x)$, or $(a, b, c) = (z, x, y)$. That is, $\sim$ is equivalence up to a cyclic rearrangement of $(a, b, c)$.

A {\em difference triple} modulo $n$ is defined to be an equivalence class of $A$ under $\sim$, and we shall use the notation $(a, b, c)$ to refer to the equivalence class containing $(a, b, c)$.
%That is, difference triples $(a, b, c)$ and $(x, y, z)$ are considered equal if $(a, b, c) \sim (x, y, z)$.
\qed
\end{definition}

\begin{definition} \label{def:complete-orbit-types}
Let $\pi$ be given by \eqref{eq:complete-permutation}, and consider the orbit of each edge of $K_{v}^{(3)}$ under $\pi$.
We define {\em types} of orbits as follows:

\begin{itemize}
    \item The orbit containing all edges of the form $\{\infty_1, \infty_2, i\}$ for $i \in \Z_n$ is of type $(\infty_1, \infty_2)$;
    \item For each $1 \leq d \leq \floor{\frac{n}{2}}$ and $k \in \{1, 2\}$, the orbit containing all edges of the form $\{\infty_k, i, i+d\}$ for $i \in \Z_n$ is of type $(\infty_k, d)$;
    \item For each difference triple $(a, b, c)$ modulo $n$, the orbit containing all edges of the form $\{i, i+a, i+a+b\}$ for $i \in \Z_n$ is of type $(a, b, c)$.
\end{itemize}
\qed
\end{definition}

It is clear that the orbit of every edge of $K_{v}^{(3)}$ falls into precisely one of the types of Definition \ref{def:complete-orbit-types}.
In order to generate difference triples modulo $n$, we use the following lemma:

\begin{lemma} \label{lem:diff-triple-existence}
Let $1 \leq a,b \leq \floor{\frac{n}{2}}$ be integers.
There exists a unique (up to equivalence) difference triple of the form $(a, b, c)$, where
$a + b = c$ if and only if $a + b \leq \floor{\frac{n}{2}}$, and $a + b + c = n$ if and only if $a + b \geq \ceil{\frac{n}{2}}$.
Moreover, $(b, c, a)$ and $(c, a, b)$ are difference triples equivalent to $(a, b, c)$ if and only if $a + b + c = n$.
\end{lemma}

\begin{proof}
If $(a, b, c)$ is a difference triple, recall that $1 \leq a,b,c \leq \floor{\frac{n}{2}}$.

Let $1 \leq a,b \leq \floor{\frac{n}{2}}$ be integers.
It is then immediate that $(a, b, a+b)$ is a difference triple if and only if $a + b \leq \floor{\frac{n}{2}}$.
Let $c = n-a-b$, then $(a, b, c)$ is a difference triple with $a + b + c = n$ if and only if $c = n - a - b \leq \floor{\frac{n}{2}}$, which holds if and only if $a + b \geq n - \floor{\frac{n}{2}} = \ceil{\frac{n}{2}}$.

If $a + b + c = n$, then clearly $(b, c, a)$ and $(c, a, b)$ are difference triples equivalent to $(a, b, c)$.
Conversely, if $a + b + c \neq n$, then $a + b = c$, and since $1 \leq a,b$, $b + c = a + 2b \neq a$ and $c + a = 2a + b \neq b$, therefore $(b, c, a)$ and $(c, a, b)$ are not difference triples.
\end{proof}

Given a $3$-uniform hypergraph $H$, and $v \geq 3$, we can now describe a method of generating an $H$-decomposition of $K_{v}^{(3)}$ of the form \eqref{eq:decomposition-orbit}.

First, given $v$ and $n = |\pi| \in \{v-2, v-1, v\}$, Algorithm \ref{alg:get-complete-types} below computes the set $\mathcal{O}$ of all orbit types of $\pi$ acting on $K_{v}^{(3)}$, as given in Definition \ref{def:complete-orbit-types}.
Lemma \ref{lem:diff-triple-existence} ensures that we have indeed generated all orbit types.

\begin{algorithm}
\KwIn{$K_{v}^{(3)}$ with vertex set given by \eqref{eq:complete-vertices}, where $v \geq 3$, $n \in \{v-2, v-1, v\}$}
\KwOut{The set of all orbit types for a cyclic permutation of length $n$ on $K_v^{(3)}$, as given by Definition \ref{def:complete-orbit-types}}

$\mathcal{O} \gets \emptyset$\;

\For{$a = 1$ \KwTo $\floor{\frac{n}{2}}$}{
    \For{$b = 1$ \KwTo $\floor{\frac{n}{2}}$}{
%        \tcc{Two cases: $a + b = c$ and $a + b + c = n$.}
        \uIf{$a + b < \ceil{\frac{n}{2}}$}{
            \tcc{We have $a + b + c \neq n$.}
            $c \gets a + b$\;
            $\mathcal{O} \gets \mathcal{O} \cup \{(a, b, c)\}$\;
        }\Else{
            \tcc{We have $a + b + c = n$.}
            $c \gets n - (a + b)$\;
            \tcc{We have $(a, b, c) \sim (b, c, a) \sim (c, a, b)$. WLOG, take the lexicographically smallest one.}
            \If{$(a, b, c) \leq (b, c, a)$ {\rm and} $(a, b, c) \leq (c, a, b)$}{
                $\mathcal{O} \gets \mathcal{O} \cup \{(a, b, c)\}$\;
            }
        }
    }
}

\If{$n \leq v-1$}{
    \For{$d = 1$ \KwTo $\floor{\frac{n}{2}}$}{
        \For{$k = 1$ \KwTo $v-n$}{
            $\mathcal{O} \gets \mathcal{O} \cup \{(\infty_k, d)\}$\;
        }
    }
}

\If{$n = v-2$}{
    $\mathcal{O} \gets \mathcal{O} \cup \{(\infty_1, \infty_2)\}$\;
}

\KwRet{$\mathcal{O}$}\;

\caption{Find orbit types of $K_v^{(3)}$ under a cyclic permutation.} \label{alg:get-complete-types}
\end{algorithm}

Next, recall that in order to generate the set of base blocks $X$, we require a mapping from admissible sets of orbit types $B$ to their corresponding hypergraphs $H'$.
Algorithm \ref{alg:get-possible-blocks} below constructs a function $c$ such that for every admissible set $B$, $c(B)$ is a hypergraph corresponding to $B$.

To achieve this, we begin by letting $c$ be the `empty function', i.e. the function with domain $\Dom(c) = \emptyset$. %\footnote{For the purposes of this discussion, we will allow functions to have an empty domain.}
We consider every subhypergraph $H'$ of $K_{v}^{(3)}$ isomorphic to
$H$,\footnote{To find a subhypergraph of $K_{v}^{(3)}$ isomorphic to $H$, fix an ordering of $V(H)$, then let an $V$ be an ordered selection of $|V(H)|$ elements from $V(K_{v}^{(3)})$. Since $V(H)$ and $V$ are ordered sets of equal size, there is a natural map $\varphi: V(H) \to V$, then $\varphi(H)$ is a subhypergraph of $K_{v}^{(3)}$. Repeating this for all ordered selections $V \subseteq V(K_{v}^{(3)})$ of size $|V(H)|$ gives all subhypergraphs of $K_{v}^{(3)}$.}
and find the set $B$ consisting of the orbit types of edges in $H'$, using Algorithm \ref{alg:complete-hypergraph-type}.
If each edge of $H'$ has a different orbit, then $B$ is admissible; and if $B$ is not in the domain of $c$, then we add $B$ to the domain and set $c(B) = H'$.

\begin{algorithm}
\KwIn{A subhypergraph $H'$ of $K_{v}^{(3)}$, where $V(K_{v}^{(3)})$ is given by \eqref{eq:complete-vertices}}
\KwOut{A set $B$ consisting of the types of orbits of edges in $H'$
%(Hence if each edge of $H'$ has a different orbit, then $B$ will be admissible and $H'$ will correspond to $B$.)
}
$B \gets \emptyset$\;
\For{$e \in E(H')$} {
    \uIf{$e = \{u, v, w\}$: $u, v, w \in \Z_n$, $u < v < w$} {
        %$a \gets \min\{v-u, n-(v-u)\}$\;
        %$b \gets \min\{w-v, n-(w-v)\}$\;
        %$c \gets \min\{w-u, n-(w-u)\}$\;
        %\uIf{$a + b + c = n$} {
        %    $t \gets \min\{(a,b,c), (b,c,a), (c,a,b)\}$ \tcc*{where $\min$ gives the lexicographically smallest ordered tuple}
        %} \uElseIf {$a + b = c$} {
        %    $t \gets (a, b, c)$\;
        %} \uElseIf {$b + c = a$} {
        %    $t \gets (b, c, a)$\;
        %} \ElseIf {$c + a = b$} {
        %    $t \gets (c, a, b)$\;
        %}
        $t \gets$ $(a, b, c)$ such that $e=\{x,x+a,x+a+b\}$ for some $x \in \Z_n$\;
        $B \gets B \cup \{t\}$\;
    } \uElseIf{$e = \{\infty_k, v, w\}$: $k \in \{1,2\}$, $v, w \in \Z_n$, $v < w$} {
        $d \gets \min \{w-v, n-(w-v)\}$\;
        $B \gets B \cup \{(\infty_k, d)\}$ \;
    } \ElseIf{$e = \{\infty_1, \infty_2, w\}$: $w \in \Z_n$} {
        $B \gets B \cup \{(\infty_1, \infty_2)\}$\;
    }
}
\KwRet{$B$}\;
\caption{Finding the set of orbit types described by a subhypergraph $H'$ of $K_{v}^{(3)}$.} \label{alg:complete-hypergraph-type}
\end{algorithm}


\begin{algorithm}
\KwIn{$3$-uniform hypergraphs $H$ and $K$, where $K$ is isomorphic to one of $K_{v}^{(3)}$, $L_{m,m,[\epsilon]}^{(3)}$, or $K_{m,m,m}^{(3)}$}
\KwOut{A function $c$ which maps admissible sets of orbits to one of their corresponding hypergraphs}
\SetKwFunction{getorbits}{get\_orbits}
\SetKwFunction{isomcopy}{copy}

$c \gets$ empty mapping\;
\For{{\rm each ordered selection} $V$ {\rm of} $|V(H)|$ {\rm elements from} $V(K)$}{
    $H' \gets \isomcopy{H, V}$ \tcc*{Isomorphic copy of $H$ on $V$}
    \If{$H'$ {\rm is a subhypergraph of} $K$} {
        $B \gets \getorbits{H'}$
        \tcc*{Algorithms \ref{alg:complete-hypergraph-type}, \ref{alg:candelabra-hypergraph-type}, \ref{alg:gdd-hypergraph-type}} % TODO refs
        %\tcc{If there are $|E(H)|$ distinct orbit types in $B$, and we haven't already found a corresponding hypergraph $c(B)$, then modify $c$ to map $B \mapsto H'$.}
        \If{$|B| = |E(H)|$ {\rm and} $B \not\in \Dom(c)$} {
            Define $c(B) = H'$\;
        }
    }
}
\KwRet{$c$}\;

\caption{Finding a mapping between admissible sets of orbits and corresponding $H$-blocks.} \label{alg:get-possible-blocks}
\end{algorithm}

Finally, Algorithm \ref{alg:decompose} describes a method for finding an $H$-decomposition of $K_{v}^{(3)}$ of the form \eqref{eq:decomposition-orbit}, if one exists. It does so by employing a recursive procedure {\tt find\_partition}, which takes a set of orbit types $\mathcal{O}$, the function $c$ above, and $e = |E(H)|$.

Note that we will reuse Algorithms \ref{alg:get-possible-blocks} and \ref{alg:decompose} in Sections \ref{sec:methods/candelabra} and \ref{sec:methods/gdd} to find decompositions of $L_{m,m,[\epsilon]}^{(3)}$ and $K_{m,m,m}^{(3)}$ respectively, so the algorithms also accept a parameter $K$, the hypergraph we wish to find a decomposition of.

\begin{algorithm}
\KwIn{$3$-uniform hypergraphs $H$ and $K$, where $K$ is isomorphic to one of $K_{v}^{(3)}$, $L_{m,m,[\epsilon]}^{(3)}$, or $K_{m,m,m}^{(3)}$}
\KwOut{The order $n$ of the cyclic permutation $\pi$, and a set $X$ of base blocks}
\SetKwFunction{gettypes}{get\_types}
\SetKwFunction{getadmissible}{get\_admissible}
\SetKwFunction{findpartition}{find\_partition}
\SetKwProg{procedure}{Procedure}{}{}

%$n \gets \max(\{n \in \{v-2, v-1, v\} : n \mid {v \choose 3}/|E(H)|\})$\;
$\mathcal{O} \gets$ \gettypes{$K$} \tcc*{Algorithms \ref{alg:get-complete-types}, \ref{alg:get-candelabra-types}, \ref{alg:get-gdd-types}}
$c \gets$ \getadmissible{$H$, $K$} \tcc*{Algorithm \ref{alg:get-possible-blocks}}
$X \gets$ \findpartition{$\mathcal{O}$, $c$, $e|E(H)|$} \tcc*{See below}
\KwRet{$n$, $X$}

\procedure{\findpartition{$\mathcal{O}$, $c$, $e$}}{
  \tcc{This recursive procedure will attempt to find a partition of $\mathcal{O}$ into admissible parts.}
  \If{$\mathcal{O} = \emptyset$}{
    \KwRet{$\emptyset$}\;
  }
  \For{{\rm each} $B \subseteq \mathcal{O}$, $|B| = e$}{
    \If{$B \in \Dom(c)$}{
        $X \gets$ \findpartition{$\mathcal{O} \setminus B$, $c$, $e$} \;
        \If{$X \neq$ ``No Solution''}{
            $X \gets X \cup \{c(B)\}$ \;
            \KwRet{$X$}\;
        }
    }
  }
  \KwRet{``No Solution''}\;
}

\caption{Finding $H$-decompositions of a hypergraph $K$} \label{alg:decompose}

\end{algorithm}

% TODO relocate this.
% These three algorithms have been implemented in Appendix \ref{appendix:code}, and were used to construct Examples \ref{eg:H_3i-9}, \ref{eg:H_3i-10}, and \ref{eg:H_3i-11}.

\newpage
