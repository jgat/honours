\section{Definitions}

The study of hypergraph decompositions is related to the study of $t$-wise balanced designs, so there are shared definitions:

\begin{definition}
A $t$-$(v,K,\lambda)$ design (also called a {\em $t$-wise balanced design}) is a pair $(X, \mathcal{B})$ where $X$ is a set of $v$ {\em points} and $\mathcal{B}$ is a collection of subsets of $X$, called {\em blocks}, such that for each $B \in \mathcal{B}$, $|B| \in K$, and every $t$-subset of $X$ is contained in exactly $\lambda$ blocks.
If $K$ has only one element $k$, then we write $t$-$(v, k, \lambda)$, instead of $t$-$(v,\{k\},\lambda)$.
\qed
\end{definition}

A $t$-$(v, k, 1)$ design can also be referred to as a {\em Steiner system} $S(t, k, v)$.
A design is {\em simple} if no two of its blocks are identical.

There is a natural correspondence between simple $t$-designs and $K_{k}^{(t)}$-decompositions of $\lambda K_{v}^{(t)}$: given a $t$-$(v, k, \lambda)$ design $(X, \mathcal{B})$, consider the collection of hypergraphs $\mathcal{D} = \{K_{B}^{(t)} \mid B \in \mathcal{B}\}$ formed by taking the complete $t$-uniform hypergraph on each block $B$.
Since each edge of $K_v^{(t)}$ occurs $\lambda$ times as an edge in some hypergraph of $\mathcal{D}$, it follows that $\mathcal{D}$ is a $K_{k}^{(t)}$-decomposition of $\lambda K_{v}^{(t)}$.
In the same manner, we can construct a $t$-$(v, k, \lambda)$ design from a $K_{k}^{(t)}$-decomposition of $\lambda K_{v}^{(t)}$, in which the vertex set of each copy of $K_k^{(t)}$ forms a block of the design.

We can then generalise this concept to other decompositions of $\lambda K_{v}^{(t)}$:

\begin{definition}
If $H$ is a $t$-uniform hypergraph, an {\em $H$-design} of order $v$ and index $\lambda$ is an $H$-decomposition of $\lambda K_{v}^{(t)}$.
Each hypergraph in the decomposition is called an {\em $H$-block}.
Such a design is sometimes denoted a $(v, H, \lambda)$-design.
\qed
\end{definition}

Unless otherwise stated, it is standard to assume that an $H$-design has index $\lambda = 1$, and we may say that the $H$-design is an $S(t, H, v)$ Steiner system.
An $H$-design is {\em simple} if no two of its $H$-blocks are identical.

If $H$ is a $t$-uniform hypergraph, the {\em complete $H$-design of order $v$} is defined to be the set of all subhypergraphs of $K_{v}^{(t)}$ which are isomorphic to $H$.

%\begin{lemma}
%The complete $H$-design of order $v$ is a simple $H$-design containing ${v \choose |V(H)|} \frac{|V(H)|!}{|\Aut(H)|}$ blocks,
%where $\Aut(H)$ is the automorphism group of $H$.
%Moreover, if $\lambda$ is the index of the complete $H$-design, and $\lambda' > \lambda$, then any $(v, H, \lambda')$-design is not simple.
%\end{lemma}

%\begin{proof}
%We must show that each edge of $K_{v}^{(t)}$ occurs in $\lambda$ subhypergraphs which are isomorphic to $H$.
%\end{proof}

The following definitions of group-divisible $t$-designs and candelabra $t$-systems are given by Moh\`{a}scy and Ray-Chaudhuri in \cite{mohascy}, where they are used to construct $S(3, k, v)$ Steiner systems.
Both of these can be considered generalisations of group divisible designs (as defined in \cite{mull-gron}) to $t \geq 3$.

\begin{definition}[\cite{mohascy}]
A {\em group-divisible $t$-design}, or $t$-GDD of order $v$, index $\lambda$, and block sizes from $K$ is a triple $(X, \mathcal{G}, \mathcal{B})$ where $X$ is a set of $v$ {\em points}, $\mathcal{G}$ is a partition of $X$ into {\em groups}, and $\mathcal{B}$ is a collection of subsets of $X$, called {\em blocks}, such that:
\begin{enumerate}
    \item[(1)] for each $B \in \mathcal{B}$, $|B| \in K$;
    \item[(2)] for each block $B \in \mathcal{B}$ and each group $G \in \mathcal{G}$, $|B \cap G| \leq 1$; and
    \item[(3)] every $t$-subset of $X$ taken from $t$ distinct groups is contained in exactly $\lambda$ blocks.
\end{enumerate}
\qed
\end{definition}

Let $(X, \mathcal{G}, \mathcal{B})$ be a group-divisible $t$-design of index $1$, where every block has size $k$, and denote the groups of the design by $\mathcal{G} = \{G_1, G_2, \ldots, G_m\}$.
This design corresponds to a $K_{k}^{(t)}$-decomposition of $K_{G_1, G_2, \ldots, G_m}^{(t)}$, where each block $B \in \mathcal{B}$ corresponds to the block $K_{B}^{(t)}$ of the decomposition.
%Conversely, every $K_{k}^{(t)}$-decomposition of $K_{v_1, v_2, \ldots, v_m}^{(t)}$ corresponds to a group-divisible $t$-design in the same way.
Hence, decompositions of $K_{v_1, v_2, \ldots, v_m}^{(t)}$ can be considered generalisations of group divisible $t$-designs.

\begin{definition}[\cite{mohascy}]
A {\em candelabra $t$-system}, or $t$-CS of order $v$, index $\lambda$ and block sizes from $K$ is an ordered tuple $(X, S, \mathcal{G}, \mathcal{B})$ where $X$ is a set of $v$ {\em points}, $S \subseteq X$ is the {\em stem} of the candelabra, $\mathcal{G}$ is a partition of $X \setminus S$ into {\em groups}, and $\mathcal{B}$ is a collection of subsets of $X$, called {\em blocks} such that
\begin{enumerate}
    \item[(1)] for each $B \in \mathcal{B}$, $|B| \in K$; and
    \item[(2)] every $t$-subset of $X$ which is not a subset of $S \cup G$ for any $G \in \mathcal{G}$ is contained in some block, and for every $G \in \mathcal{G}$, no $t$-subset of $S \cup G$ is contained in any block.
\end{enumerate}
\qed
\end{definition}

Let $(X, S, \mathcal{G}, \mathcal{B})$ be a candelabra $t$-system of index $1$, where every block has size $k$, and denote the groups of the system by $\mathcal{G} = \{G_1, \ldots, G_m\}$.
This design corresponds to a $K_{k}^{(t)}$-decomposition of $L_{G_1,\ldots,G_m,[S]}^{(t)}$, where each block $B \in \mathcal{B}$ corresponds to the block $K_{B}^{(t)}$ of the decomposition.
%Conversely, every $K_{k}^{(t)}$-decomposition of $L_{v_1, v_2, \ldots, v_m,[w]}^{(t)}$ corresponds to a group-divisible $t$-design in the same way.
Hence, decompositions of $L_{v_1, \ldots, v_m,[w]}^{(t)}$ can be considered generalisations of candelabra $t$-systems.
