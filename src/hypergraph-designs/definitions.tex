\section{Definitions}

The study of hypergraph decompositions is related to the study of $t$-wise balanced designs, so there are shared definitions:

\begin{definition}
A $t$-$(v,K,\lambda)$ design (also called a {\em $t$-wise balanced design}) is a pair $(X, \mathcal{B})$ where $X$ is a set of $v$ {\em points} and $\mathcal{B}$ is a collection of subsets of $X$, called {\em blocks}, such that for each $B \in \mathcal{B}$, $|B| \in K$, and every $t$-subset of $X$ is contained in exactly $\lambda$ blocks.
If $K$ has only one element $k$, then we write $t$-$(v, k, \lambda)$, instead of $t$-$(v,\{k\},\lambda)$.
\qed
\end{definition}

A $t$-$(v, k, 1)$ design can also be referred to as a {\em Steiner system} $S(t, k, v)$.
A design is {\em simple} if no two of its blocks are identical.

There is a natural correspondence between simple $t$-designs and $K_{k}^{(t)}$-decompositions of $\lambda K_{v}^{(t)}$: given a $t$-$(v, k, \lambda)$ design $(X, \mathcal{B})$, consider the collection of hypergraphs $\mathcal{D} = \{K_{B}^{(t)} \mid B \in \mathcal{B}\}$ formed by taking the complete $t$-uniform hypergraph on each block $B$.
Since each edge of $K_v^{(t)}$ occurs $\lambda$ times as an edge in some hypergraph of $\mathcal{D}$, it follows that $\mathcal{D}$ is a $K_{k}^{(t)}$-decomposition of $\lambda K_{v}^{(t)}$.
In the same manner, we can construct a $t$-$(v, k, \lambda)$ design from a $K_{k}^{(t)}$-decomposition of $\lambda K_{v}^{(t)}$, in which the vertex set of each copy of $K_k^{(t)}$ forms a block of the design.

We can then generalise this concept to other decompositions of $\lambda K_{v}^{(t)}$:

\begin{definition}
If $H$ is a $t$-uniform hypergraph, an {\em $H$-design} of order $v$ and index $\lambda$ is an $H$-decomposition of $\lambda K_{v}^{(t)}$.
Each hypergraph in the decomposition is called an {\em $H$-block}.
Such a design is sometimes denoted a $(v, H, \lambda)$-design.
\qed
\end{definition}

Unless otherwise stated, it is standard to assume that an $H$-design has index $\lambda = 1$, and we may say that the $H$-design is an $S(t, H, v)$ Steiner system.
An $H$-design is {\em simple} if no two of its $H$-blocks are identical.

If $H$ is a $t$-uniform hypergraph, the {\em complete $H$-design of order $v$} is defined to be the set of all subhypergraphs of $K_{v}^{(t)}$ which are isomorphic to $H$.

%\begin{lemma}
%The complete $H$-design of order $v$ is a simple $H$-design containing ${v \choose |V(H)|} \frac{|V(H)|!}{|\Aut(H)|}$ blocks,
%where $\Aut(H)$ is the automorphism group of $H$.
%Moreover, if $\lambda$ is the index of the complete $H$-design, and $\lambda' > \lambda$, then any $(v, H, \lambda')$-design is not simple.
%\end{lemma}

%\begin{proof}
%We must show that each edge of $K_{v}^{(t)}$ occurs in $\lambda$ subhypergraphs which are isomorphic to $H$.
%\end{proof}

If there exists a $(v, H, \lambda)$-design, then there are certain necessary divisiblity conditions to be satisfied.
The following proof is from \cite{bryant}:

\begin{lemma}[\cite{bryant}] \label{obv-necessary-1}
Let $H$ and $K$ be hypergraphs. Suppose there exists an $H$-decomposition of $K$.
Then, for any subset $R$ of $V(K)$ with $|R| \leq |V(H)|$,
$\gcd(\{\deg_H(S) : S \subseteq V(H), |S| = |R|\})$ divides $\deg_K(R)$.
\end{lemma}

(If necessary, we define $\gcd(\{0\}) = 0$ in the above lemma.)

\begin{proof}
Let $\mathcal{D}$ be an $H$-decomposition of $K$, $R \subseteq V(K)$, and suppose that $|R| \leq |V(H)|$.
The $H$-blocks in $\mathcal{D}$ whose vertex sets contain $R$ partition the edges of $K$ which contain $R$, so
\begin{equation} \label{eq:degree-sum}
    \deg_K(R) = \sum_{\substack{G \in \mathcal{D} \\ R \subseteq V(G)}} \deg_G(R).
\end{equation}
For each $G \in \mathcal{D}$ with $R \subseteq V(G)$, we have
\[
    \deg_G(R) \in \{\deg_H(S) : S \subseteq V(H), |S| = |R|\}.
\]
Therefore $\gcd(\{\deg_H(S) : S \subseteq V(H), |S| = |R|\})$ divides $\deg_K(R)$, by B\'{e}zout's identity.
\end{proof}

Informally, for any set $R$ of vertices in $K$, there are $\deg_K(R)$ edges incident with $R$,
but each of these edges must occur precisely once in the design $\mathcal{D}$.
Hence, for every $G \in \mathcal{D}$ which contains $R$, the isomorphism $G \simeq H$
carries $R$ into some $S \subseteq V(H)$ with $|S| = |R|$, and so $G$ contains
$\deg_H(S)$ of the $\deg_K(R)$ edges incident with $R$.
Therefore, $\deg_K(R)$ must be given by a sum of elements in $\{\deg_H(S) : S \subseteq V(H), |S| = |R|\}$.
So, if $\deg_H(S)$ is divisible by some $d$ for all such $S$, then $\deg_K(R)$ must also be divisible by $d$.

Note that if $|R| > |V(H)|$, then there are no $G \in \mathcal{D}$ with $R \subseteq V(G)$,
so \eqref{eq:degree-sum} is the empty sum (and so $\deg_K(R) = 0$), and the statement following holds vacuously.
However, $\gcd(\{\deg_H(S) : S \subseteq V(H), |S| = |R|\}) = \gcd(\emptyset)$ is not defined,
so the conclusion does not follow.

If we take $K = \lambda K_v^{(t)}$, then we have:

\begin{lemma} \label{obv-necessary}
Let $H$ be a $t$-uniform hypergraph.
If there exists a $(v, H, \lambda)$-design, then for each $0 \leq i < t$,
  $\gcd(\{ \deg_H(S) : S \subseteq V(H), |S| = i \})$ divides $\lambda {v-i \choose t-i}$.
\end{lemma}

\begin{proof}
Apply Lemma \ref{obv-necessary-1} with $K = \lambda K_v^{(t)}$;
  note that for any $R \subseteq V(K)$ with $|R| = i$, $\deg_K(R) = \lambda {v-i \choose t-i}$.
\end{proof}

For example, if we take $H = K_k^{(t)}$, then we have the usual necessary divisibility conditions for $t$-designs: for each $0 \leq i < t$, ${k-i \choose t-i}$ divides $\lambda {v-i \choose t-i}$.

If $v < t$, then $\lambda K_v^{(t)}$ contains no edges, and we will consider the empty set to be an $H$-decomposition of $\lambda K_v^{(t)}$, and therefore a $(v, H, \lambda)$-design, for any $H$ and $\lambda$. Otherwise, if $v \geq t$ and $\mathcal{D} \neq \emptyset$ is a $H$-design, then it is necessary that there is a subgraph of $\lambda K_v^{(t)}$ isomorphic to $H$, so $|V(H)| \leq |V(\lambda K_v^{(t)})|$, hence:

\begin{lemma} \label{obv-order}
Let $H$ be a $t$-uniform hypergraph. If $\mathcal{D}$ is a $(v, H, \lambda)$-design, then either $v < t$ and $\mathcal{D} = \emptyset$ or $v \geq |V(H)|$ and $\mathcal{D} \neq \emptyset$.
\end{lemma}

Since $v < t$ is trivial, we shall assume $v \geq t$ unless otherwise stated.

The following necessary and sufficient condition follows immediately from Baranyai's Theorem in \cite{baran}. The necessary condition is immediate from Lemmas \ref{obv-necessary} and \ref{obv-order}, and Baranyai shows it is sufficient.

\begin{theorem}[\cite{baran}] \label{baranyai}
Let $H$ be the simple $t$-uniform hypergraph consisting of $m$ pairwise disjoint edges. There is an $H$-design of order $v$ if and only if $m$ divides ${v \choose t}$ and $mt \leq v$.
\end{theorem}
