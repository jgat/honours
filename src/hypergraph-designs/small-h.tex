\section{Small $3$-uniform hypergraphs} \label{section:small-h}

A central problem in the study of hypergraph decompositions is the existence of
$H$-designs; namely: given a $t$-uniform hypergraph $H$, for which positive
integers $v$ does there exist a $(v, H, \lambda)$-design?

In this section, we will consider this question for $3$-uniform hypergraphs $H$ of bounded size.
If we apply Lemmas \ref{obv-necessary} and \ref{obv-order} with $t = 3$, then we have:

\begin{lemma}[\cite{bryant}] \label{necessary-3-uniform}
Let $H$ be a $3$-uniform hypergraph.
If there exists an $H$-design of order $v \geq 3$, then $v \geq |V(H)|$, and the following conditions are satisfied:
\begin{enumerate}
    \item[(1)] $|E(H)|$ divides ${v \choose 3}$,
    \item[(2)] $\gcd(\{\deg_H(x) : x \in V(H)\})$ divides ${v-1 \choose 2}$,
    \item[(3)] $\gcd(\{\deg_H(\{x,y\}) : x, y \in V(H), x \neq y\})$ divides $v-2$.
\end{enumerate}
\end{lemma}

\subsection{$3$-uniform hypergraphs of size at most $3$}

First, we shall consider simple $3$-uniform hypergraphs containing at most three edges, by considering every possible case.
In combination with results from \cite{baran, bryant, feng-chang2, hanani}, we shall prove the following result:

\begin{theorem} \label{thm:size-le3}
Let $H$ be a simple $3$-uniform hypergraph containing at most three edges.
There exists an $H$-design of order $v \geq 3$ iff the conditions of Lemma \ref{necessary-3-uniform} are satisfied.
\end{theorem}

Up to isomorphism, there are fifteen (simple) $3$-uniform hypergraphs with size at most 3 containing no isolated vertices.
These are listed in Table \ref{table:size3} below, using the following notation: in the first row, $[1, 2, 3]_{H_{1,1}}$ is used to denote the hypergraph $(V, E)$ with $V = \{1, 2, 3\}$ and $E = \{\{1, 2, 3\}\}$, and $H_{1,1}$ is used to denote any hypergraph isomorphic to $[1, 2, 3]_{H_{1,1}}$.
The final column indicates that the existence of $H_{1,1}$-designs is proved in Theorem \ref{baranyai}.
The notation for the other rows is similar.

\begin{table}[h]
\centering
\scriptsize
\begin{tabular}{|c|c|c|c|c|}
\hline
$H_{1,1}$ & $[1,2,3]_{H_{1,1}}$            & $V=\{1,2,3\}$            & $E=\{\{1,2,3\}\}$ & \ref{baranyai} \\ \hline % TODO [6pt]
$H_{2,1}$ & $[1,2,3,4,5,6]_{H_{2,1}}$      & $V=\{1,2,3,4,5,6\}$      & $E=\{\{1,2,3\},\{4,5,6\}\}$ & \ref{baranyai} \\ \hline
$H_{2,2}$ & $[1,2,3,4,5]_{H_{2,2}}$        & $V=\{1,2,3,4,5\}$        & $E=\{\{1,2,3\},\{1,4,5\}\}$ & \ref{thm:H_22,H_23} \\ \hline
$H_{2,3}$ & $[1,2,3,4]_{H_{2,3}}$          & $V=\{1,2,3,4\}$          & $E=\{\{1,2,3\},\{1,2,4\}\}$ & \ref{thm:H_22,H_23} \\ \hline
$H_{3,1}$ & $[1,2,3,4,5,6,7,8,9]_{H_{3,1}}$& $V=\{1,2,3,4,5,6,7,8,9\}$& $E=\{\{1,2,3\},\{4,5,6\},\{7,8,9\}\}$ & \ref{baranyai} \\ \hline
$H_{3,2}$ & $[1,2,3,4,5,6,7,8]_{H_{3,2}}$  & $V=\{1,2,3,4,5,6,7,8\}$  & $E=\{\{1,2,3\},\{1,4,5\},\{6,7,8\}\}$ & \ref{thm:H_3,i} \\ \hline
$H_{3,3}$ & $[1,2,3,4,5,6,7]_{H_{3,3}}$    & $V=\{1,2,3,4,5,6,7\}$    & $E=\{\{1,2,3\},\{1,2,4\},\{5,6,7\}\}$ & \ref{thm:H_3,i} \\ \hline
$H_{3,4}$ & $[1,2,3,4,5,6,7]_{H_{3,4}}$    & $V=\{1,2,3,4,5,6,7\}$    & $E=\{\{1,2,3\},\{1,4,5\},\{1,6,7\}\}$ & \ref{thm:H_3,i} \\ \hline
$H_{3,5}$ & $[1,2,3,4,5,6,7]_{H_{3,5}}$    & $V=\{1,2,3,4,5,6,7\}$    & $E=\{\{1,2,3\},\{1,4,5\},\{4,6,7\}\}$ & \ref{thm:H_3,i} \\ \hline
$H_{3,6}$ & $[1,2,3,4,5,6]_{H_{3,6}}$      & $V=\{1,2,3,4,5,6\}$      & $E=\{\{1,2,3\},\{1,2,4\},\{4,5,6\}\}$ & \ref{thm:H_3,i} \\ \hline
$H_{3,7}$ & $[1,2,3,4,5,6]_{H_{3,7}}$      & $V=\{1,2,3,4,5,6\}$      & $E=\{\{1,2,3\},\{1,2,4\},\{2,5,6\}\}$ & \ref{thm:H_3,i} \\ \hline
$H_{3,8}$ & $[1,2,3,4,5,6]_{H_{3,8}}$      & $V=\{1,2,3,4,5,6\}$      & $E=\{\{1,2,3\},\{1,4,5\},\{2,4,6\}\}$ & \ref{thm:H_3,i} \\ \hline
$H_{3,9}$ & $[1,2,3,4,5]_{H_{3,9}}$        & $V=\{1,2,3,4,5\}$        & $E=\{\{1,2,3\},\{1,2,4\},\{1,2,5\}\}$ & \ref{thm:H_3,i} \\ \hline
$H_{3,10}$ & $[1,2,3,4,5]_{H_{3,10}}$      & $V=\{1,2,3,4,5\}$        & $E=\{\{1,2,3\},\{1,2,4\},\{1,4,5\}\}$ & \ref{thm:H_3,i} \\ \hline
$H_{3,11}$ & $[1,2,3,4,5]_{H_{3,11}}$      & $V=\{1,2,3,4,5\}$        & $E=\{\{1,2,3\},\{1,2,4\},\{3,4,5\}\}$ & \ref{thm:H_3,i} \\ \hline
$H_{3,12}$ & $[1,2,3,4]_{H_{3,12}}$        & $V=\{1,2,3,4\}$          & $E=\{\{1,2,3\},\{1,2,4\},\{1,3,4\}\}$ & \ref{thm:H_3,12} \\ \hline
\end{tabular}
\caption{All simple $3$-uniform hypergraphs of size at most $3$}
\label{table:size3}
\end{table}

It can be verified exhaustively that these are indeed all simple $3$-uniform hypergraphs of size at most 3.
% TODO write up the program which generates them.

If $H \simeq H_{k,1}$ for $k \in \{1, 2, 3\}$, then $H$ contains $k$ pairwise disjoint edges, so $H$-designs exist iff $v \geq 3k$ and $k$ divides ${v \choose 3}$, by Theorem \ref{baranyai}.

In \cite{feng-chang2}, the authors consider a copy of $K_4^{(3)}$ with an edge removed, denoted by $K_4^{(3)} - e$; this is isomorphic to $H_{3,12}$. They determined the following result:

\begin{theorem}[\cite{feng-chang2}] \label{thm:H_3,12}
A $(K_4^{(3)}-e)$-design of order $v$ exists iff $v \equiv 0$,$1$, or $2$ (mod $9$).
\end{theorem}

When $H$ has size $2$, the remaining cases were solved by Bryant et al., and we describe the proof below.

\begin{theorem}[\cite{bryant}] \label{thm:H_22,H_23}
Let $H \simeq H_{2,2}$ or $H_{2,3}$.
An $H$-design of order $v$ exists if and only if $v \equiv 0$, $1$, or $2$ (mod $4$) and $v \geq |V(H)|$.
\end{theorem}

Since each of $H_{2,2}$ and $H_{2,3}$ contains at least one vertex of degree 1, the conditions of Lemma \ref{necessary-3-uniform}
reduce to $v \geq |V(H)|$ and $|E(H)| = 2$ divides ${v \choose 3}$. This second condition is equivalent to $v \equiv 0$, $1$, or $2$ (mod $4$).

To construct an $H$-design of order $v = 4n + \epsilon$, where $\epsilon \in \{0,1,2\}$ and $n \geq 1$
(excluding the case $H \simeq H_{2,2}$ and $v = 4$), construct a collection of pairwise disjoint sets
$V_1, V_2, \ldots, V_n$, each of size 4. Let $\infty_1, \infty_2$ be two points not contained in
$\bigcup_{i=1}^{n} V_i$, and let $I_0 = \emptyset$, $I_1 = \{\infty_1\}$, $I_2 = \{\infty_1, \infty_2\}$.
We then require an $H$-decomposition of $K = K_{V_1 \cup V_2 \cup \cdots \cup V_n \cup I_\epsilon}^{(3)} \simeq K_{v}^{(3)}$.

For each edge $e \in E(K)$, consider the sets $V_i$ which have non-empty intersection with $e$:
\begin{itemize}
    \item If $e \subseteq V_i \cup I_\epsilon$ for some $i \in \{1,\ldots,n\}$, then $e$ is an edge of $K_{V_i \cup I_\epsilon}^{(3)} \simeq K_{4+\epsilon}^{(3)}$.
    \item If $e \subseteq V_i \cup V_j \cup I_\epsilon$ for distinct $i, j \in \{1,\ldots,n\}$, and has at least one vertex in each of $V_i$ and $V_j$, then $e$ is an edge of $L_{V_i,V_j,[I_\epsilon]}^{(3)} \simeq L_{4,4,[\epsilon]}^{(3)}$.
    \item If $e \subseteq V_i \cup V_j \cup V_k$ for distinct $i, j, k \in \{1,\ldots,n\}$, and has at least one vertex in each of $V_i$, $V_j$, and $V_k$, then $e$ is an edge of $K_{V_i,V_j,V_k}^{(3)} \simeq K_{4,4,4}^{(3)}$.
\end{itemize}
Therefore,
\[
    K
    %= K_{V_1 \cup \cdots \cup V_n \cup I_\epsilon}^{(3)}
    = \left( \bigcup_{1 \leq i \leq n} K_{V_i \cup I_\epsilon}^{(3)} \right)
    \cup \left( \bigcup_{1 \leq i < j \leq n} L_{V_i,V_j,[I_\epsilon]}^{(3)} \right)
    \cup \left( \bigcup_{1 \leq i < j < k \leq n} K_{V_i,V_j,V_k}^{(3)} \right)
\]
is a $\Gamma_\epsilon$-decomposition of $K$, where
\[
    \Gamma_\epsilon = \left\{ K_{4+\epsilon}^{(3)}, L_{4,4,[\epsilon]}^{(3)}, K_{4,4,4}^{(3)} \right\}.
\]
Hence, if we can find an $H$-decomposition of each $G \in \Gamma_\epsilon$, this can be extended to an $H$-decomposition of $K$, as required.
Examples of $H$-decompositions of $G$ for each $H \in \left\{ H_{2,2}, H_{2,3} \right\}$
and $G \in \left\{ K_{4}^{(3)}, K_{5}^{(3)}, K_{6}^{(3)}, L_{4,4}^{(3)}, L_{4,4,[1]}^{(3)}, L_{4,4,[2]}^{(3)}, K_{4,4,4}^{(3)} \right\}$
are given in \cite{bryant}, except when $H \simeq H_{2,2}$ and $G \simeq K_{4}^{(3)}$,
so this completes the proof aside from the case $H \simeq H_{2,2}$ and $v \equiv 0$ (mod $4$), $v \geq 8$.

In this remaining case, one can similarly find a $\Gamma'$-decomposition of $K$ where
\[
    \Gamma' = \left\{ K_{8}^{(3)}, K_{8}^{(3)} \setminus K_{4}^{(3)}, L_{4,4}^{(3)}, K_{4,4,4}^{(3)} \right\},
\]
given by
\begin{align*}
    K &= K_{V_{n-1} \cup V_n}^{(3)}
    \cup \left( \bigcup_{1 \leq i \leq n-1} K_{V_i \cup V_n}^{(3)} \setminus K_{V_n}^{(3)} \right) \\
    & \quad \quad \cup \left( \bigcup_{1 \leq i < j \leq n-1} L_{V_i, V_j}^{(3)} \right)
    \cup \left( \bigcup_{1 \leq i < j < k \leq n} K_{V_i, V_j, V_k}^{(3)} \right).
\end{align*}
It can be shown that there exists an $H_{2,3}$-decomposition of $G$ for each $G \in \Gamma'$, so this extends to
an $H_{2,3}$-decomposition of $v = 4n$.

We use a similar method to prove Theorems \ref{thm:H_3,i} and \ref{thm:H_42,H_43} below.

\begin{theorem} \label{thm:H_3,i}
Let $H$ be a simple $3$-uniform hypergraph with three edges.
There exists an $H$-design of order $v$ if and only if $v \equiv 0$, $1$, or $2$ (mod $9$).
\end{theorem}

The case where $H$ has five or six vertices has been solved in \cite{bryant},
so it remains to prove the result when $H \simeq H_{3,2}$, $H_{3,3}$, $H_{3,4}$, or $H_{3,5}$.

\begin{proof}
Suppose that $H$ is isomorphic to one of $H_{3,2}$, $H_{3,3}$, $H_{3,4}$, or $H_{3,5}$.
From Lemma \ref{necessary-3-uniform}, if an $H$-design of order $v$ exists, then $v \equiv 0$, $1$, or $2$ (mod $9$), it remains to show this is sufficient.

Suppose that $v = 9n + \epsilon$ for $n \geq 1$ and $\epsilon \in \{0,1,2\}$.
Construct $n$ pairwise disjoint sets $V_1, \ldots, V_n$ of size 9,
and let $\infty_1, \infty_2$ be two distinct points, neither of which is in $\bigcup_{i=1}^{n} V_i$.
Let $I_0 = \emptyset$, $I_1 = \{\infty_1\}$ and $I_2 = \{\infty_1, \infty_2\}$.

For each $1 \leq i \leq n$ and each $\epsilon \in \{0,1,2\}$, let $\mathcal{D}_i^{\epsilon}$ be an $H$-decomposition of $K_{V_i \cup I_\epsilon}^{(3)}$,
which exists by Examples \ref{eg:H_3i-9}, \ref{eg:H_3i-10}, and \ref{eg:H_3i-11} for $\epsilon = 0$, $1$, and $2$ respectively.
%
For each $1 \leq i < j \leq n$ and each $\epsilon \in \{0,1,2\}$, let $\mathcal{D}_{i,j}^{\epsilon}$ be an $H$-decomposition of $L_{V_i, V_j, [I_\epsilon]}^{(3)}$, which exists by Examples \ref{eg:H_3i-l99}, \ref{eg:H_3i-l99-k991}, and \ref{eg:H_3i-l99-k992} for $\epsilon = 0$, $1$, and $2$ respectively.
%
For each $1 \leq i < j < k \leq n$, let $\mathcal{D}_{i,j,k}$ be an $H$-decomposition of $K_{V_i,V_j,V_k}^{(3)}$, which exists by Example \ref{eg:H_3i-k999}.

Then,
\[
    \mathcal{D} = \left( \bigcup_{1 \leq i \leq n} \mathcal{D}_{i}^{\epsilon} \right)
    \cup \left( \bigcup_{1 \leq i < j \leq n} \mathcal{D}_{i,j}^{\epsilon} \right)
    \cup \left( \bigcup_{1 \leq i < j < k \leq n} \mathcal{D}_{i,j,k}^{\epsilon} \right)
\]
is an $H$-design of order $v = 9n + \epsilon$, for each $n \geq 1$ and $\epsilon \in \{0,1,2\}$.
\end{proof}

%To see why $\mathcal{D}$ is an $H$-design of order $9n + \epsilon$, consider any edge $e$ of $K_{V_1 \cup \cdots \cup V_n \cup I_\epsilon}^{(3)}$, and consider which sets $V_i$ contain a vertex in $e$.
%If all vertices of $e$ are in $V_i \cup I_\epsilon$ for some $i$, then some $H$-block of $\mathcal{D}_i^{\epsilon}$ contains $e$.
%Otherwise, if all vertices of $e$ are in $V_i \cup V_j \cup I_\epsilon$ for some distinct $i, j$, then some $H$-block of $\mathcal{D}_{i,j}^{\epsilon}$ contains $e$.
%Otherwise, it must be the case that the three vertices of $e$ are in $V_i \cup V_j \cup V_k$ for some distinct $i, j, k$, so $e$ is contained in some $H$-block of $\mathcal{D}_{i,j,k}^{\epsilon}$.
%Precisely one of these three cases must hold, so every edge of $K_{V_1 \cup \cdots \cup V_n \cup I_\epsilon}^{(3)}$ is contained in precisely one $H$-block of $\mathcal{D}$.

When $H$ is isomorphic to one of $H_{3,6}$, $H_{3,7}$, $H_{3,8}$, $H_{3,9}$ or $H_{3,10}$, the proof is essentially identical, since it has been shown in \cite{bryant} that there exist $H$-decompositions of $K_{v}^{(3)}$ for each $v \in \{9,10,11\}$, $L_{9,9,[\epsilon]}^{(3)}$ for each $\epsilon \in \{0,1,2\}$, and $K_{9,9,9}^{(3)}$.

The case $H \simeq H_{3,11}$ must be treated differently, since there is no subgraph of $K_{9,9,9}^{(3)}$ isomorphic to $H_{3,11}$
(and hence there does not exist an $H_{3,11}$-decomposition of $K_{9,9,9}^{(3)}$).

Instead, the authors of \cite{bryant} aim to find a $\Gamma_\epsilon$-decomposition of the complete $3$-uniform hypergraph $K = K_{V_1 \cup \cdots \cup V_n \cup I_\epsilon}^{(3)}$, where
\[
    \Gamma_\epsilon =
    \left\{ K_{9}^{(3)}, L_{9,9,[\epsilon]}^{(3)}, L_{U,V,[I_\epsilon]}^{(3)} \cup K_{U,V,W}^{(3)}, K_{9,9,9,9}^{(3)} \right\},
\]
where $U$, $V$, $W$, $I_\epsilon$ are pairwise disjoint sets with $|U| = |V| = |W| = 9$ and $|I_\epsilon| = \epsilon$.

To do this, define $K_{\{1,\ldots,n\}}^{(2,3)} = K_{\{1,\ldots,n\}}^{(2)} \cup K_{\{1,\ldots,n\}}^{(3)}$; that is, $V(K_{\{1,\ldots,n\}}^{(2,3)}) = \{1,2,\ldots,n\}$ and $e$ is an edge iff $|e| \in \{2,3\}$.
It can be shown that there exists a $\Lambda$-decomposition $\mathcal{D}'$ of $K_{\{1,\ldots,n\}}^{(2,3)}$,
where $\Lambda = \left\{ K_{2}^{(2)}, [1,2,3]_A, K_{4}^{(3)} \right\}$, and where $[1,2,3]_A$ is the hypergraph $(\{1,2,3\}, \{\{1,2\},\{1,2,3\}\})$.
So, $\mathcal{D}'$ gives rise to a $\Gamma_\epsilon$-decomposition of $K$ given by
\begin{align*}
    K &= \left( \bigcup_{1 \leq i \leq n} K_{V_i \cup I_\epsilon}^{(3)} \right)
    \cup \left( \bigcup_{K_{\{i,j\}}^{(2)} \in \mathcal{D}'} L_{V_i,V_j,[I_\epsilon]}^{(3)} \right)
    \\ & \quad \quad
    \cup \left( \bigcup_{[i,j,k]_A \in \mathcal{D}'} L_{V_i,V_j,[I_\epsilon]}^{(3)} \cup K_{V_i,V_j,V_k}^{(3)} \right)
    \cup \left( \bigcup_{K_{\{i,j,k,l\}}^{(3)} \in \mathcal{D}'} K_{V_i,V_j,V_k,V_l}^{(3)} \right),
\end{align*}
and simlarly for $\epsilon \in \{1,2\}$.
Informally, $\mathcal{D}'$ describes a pattern of arranging copies of $L \in \Lambda$ to give every pair and triple of points from the set $\{1,2,\ldots,n\}$,
so we can therefore cover every $L_{V_i, V_j, [I_\epsilon]}^{(3)}$ and every $K_{V_i, V_j, V_k}^{(3)}$ using this pattern.

For each $\epsilon \in \{0,1,2\}$ and each $G \in \Gamma_\epsilon$, it can be shown that there exists an $H_{3,11}$-decomposition of $G$,
so we can construct an $H_{3,11}$-decomposition of $K$, which completes the proof.

This completes the proof of Theorem \ref{thm:size-le3}. \qed

\subsection{Regular $3$-uniform hypergraphs}

Next, we shall consider (simple) $3$-uniform hypergraphs of larger size, but impose the restriction that the hypergraph is regular.
Recall that a hypergraph $H$ is $d$-regular if every vertex $x \in V(H)$ has degree $d$.

There are nine non-isomorphic regular $3$-uniform hypergraphs of size at most $5$, three of which are $H_{1,1}$, $H_{2,1}$ and $H_{3,1}$ from Table \ref{table:size3}.
The remaining six are listed in Table \ref{table:regular}, using the same notation as Table \ref{table:size3}.

\begin{table}[h]
\centering
\scriptsize
\begin{tabular}{|c|c|c|c|c|}
\hline
$H_{4,1}$ & $[1,2,\ldots,12]_{H_{4,1}}$ & $V=\{1,2,\ldots,12\}$ & $E=\{\{1,2,3\},\{4,5,6\},\{7,8,9\},\{10,11,12\}\}$ & \ref{baranyai} \\ \hline
$H_{4,2}$ & $[1,2,3,4,5,6]_{H_{4,2}}$   & $V=\{1,2,3,4,5,6\}$ & $E=\{\{1,2,3\},\{1,5,6\},\{2,4,6\},\{3,5,6\}\}$ & \ref{thm:H_42,H_43} \\ \hline
$H_{4,3}$ & $[1,2,3,4,5,6]_{H_{4,3}}$   & $V=\{1,2,3,4,5,6\}$ & $E=\{\{1,2,3\},\{1,5,6\},\{2,3,4\},\{4,5,6\}\}$ & \ref{thm:H_42,H_43} \\ \hline
$H_{4,4}$ & $[1,2,3,4]_{H_{4,4}}$       & $V=\{1,2,3,4\}$     & $E=\{\{1,2,3\},\{1,2,4\},\{1,3,4\},\{2,3,4\}\}$ & \ref{thm:K4^3} \\ \hline
$H_{5,1}$ & $[1,2,\ldots,15]_{H_{5,1}}$ & $V=\{1,2,\ldots,15\}$ & {\tiny $\{\{1,2,3\},\{4,5,6\},\{7,8,9\},\{10,11,12\},\{13,14,15\}\}$ } & \ref{baranyai} \\ \hline
$H_{5,2}$ & $[1,2,3,4,5]_{H_{5,2}}$     & $V=\{1,2,3,4,5\}$   & {\tiny $E=\{\{1,2,3\},\{1,2,4\},\{1,3,5\},\{2,4,5\},\{3,4,5\}\}$ } & Open \\ \hline
\end{tabular}
\caption{All simple $d$-regular $3$-uniform hypergraphs with size $\in \{4, 5\}$}
\label{table:regular}
\end{table}

It can be easily verified that these are indeed all regular $3$-uniform hypergraphs with size at most 5. % TODO algorithm
In conjunction with existing results, we shall prove the following:

\begin{theorem} \label{thm:regular-4}
Let $H$ be a simple regular $3$-uniform hypergraph containing four edges.
There exists an $H$-design of order $v \geq 3$ iff the conditions of Lemma \ref{necessary-3-uniform} are satisfied,
  except that there is no $H_{4,2}$-design of order $6$.
\end{theorem}

Let $H$ be a $d$-regular $3$-uniform hypergraph on four edges.
If $H \simeq H_{4,1}$ or $H_{5,1}$, the necessary conditions are sufficient by Theorem \ref{baranyai}.

If $H \simeq H_{4,4} \simeq K_4^{(3)}$, then an $H$-design of order $v$ is an $S(3, 4, v)$ Steiner system (also called a {\em Steiner quadruple system}),
  and it can be seen that $v$ satisfies the conditions of Lemma \ref{necessary-3-uniform} if and only if $v \equiv 2$ or $4$ (mod $6$).
The following result by Hanani establishes that these conditions are sufficient:

\begin{theorem}[\cite{hanani}] \label{thm:K4^3}
An $S(3, 4, v)$ exists iff $v \equiv 2$ or $4$ (mod $6$).
\end{theorem}


For $H \simeq H_{4,2}$ or $H_{4,3}$, we have the following theorem.
The case $H \simeq H_{4,2}$ was solved by Bryant et al. in \cite{bryant}, and the proof for $H \simeq H_{4,3}$ is similar.

\begin{theorem} \label{thm:H_42,H_43}
Let $H \simeq H_{4,2}$ or $H_{4,3}$.
An $H$-design of order $v$ exists iff $v \equiv 1$, $2$ or $6$ (mod $8$),
  except that there is no $H_{4,2}$-design of order $6$.
\end{theorem}

\begin{proof}
Let $H$ be a hypergraph isomorphic to either $H_{4,2}$ or $H_{4,3}$.

From Lemma \ref{necessary-3-uniform}, it follows that $v \equiv 1$, $2$ or $6$ (mod $8$) is a necessary condition,
   it remains to show that this is sufficient, and that there is no $(6, H_{4,2}, 1)$-design.

For $v = 6$, there exists an $H_{4,3}$-design of order 6 by Example \ref{eg:H_43-6},
  and it was shown in \cite{bryant} that there does not exist an $H_{4,2}$-design of order 6. So, it remains to consider $v \geq 9$.

Let $O$ denote the hypergraph with vertex set $\{0,1,2,3,4,5\}$ and edge set
\[
    \{\{0,1,2\},\{0,1,5\},\{0,2,4\},\{0,4,5\},\{1,2,3\},\{1,3,5\},\{2,3,4\},\{3,4,5\}\}.
\]
The edges of $O$ describe the faces of an octahedron.
There clearly exists an $H_{4,2}$-decomposition of $O$ given by \[ \left\{ [0,1,2,3,4,5]_{H_{4,2}}, [0,1,5,3,4,2]_{H_{4,2}} \right\}, \]
  and an $H_{4,3}$-decomposition of $O$ given by \[ \left\{[0,1,2,3,4,5]_{H_{4,3}}, [0,1,5,3,2,4]_{H_{4,3}} \right\}. \]
In \cite{hanani}, it was shown that there exists an $O$-design of order $v$ whenever $v \equiv 2$ (mod 8),
  so there exists a $H$-design of order $v$ whenever $v \equiv 2$ (mod 8).


In the remaining cases, we have $v = 8n + \epsilon$ for $n \geq 1$ and $\epsilon \in \{1,6\}$.
Construct a collection of $n$ pairwise disjoint sets $V_1, \ldots, V_n$ of size 8,
  and let $I = \{\infty_1, \infty_2, \infty_3, \infty_4, \infty_5, \infty_6\}$ be a set of six points, none of which is in $\bigcup_{i=1}^{n} V_i$.

For each $1 \leq i \leq n$:
\begin{itemize}
  \item let $\mathcal{D}_i$ be an $H$-decomposition of $K_{V_i \cup \{\infty_1\}}$, which exists for $H_{4,2}$ by \cite{bryant}, and for $H_{4,3}$ by Example \ref{eg:H_43-9}; and
  \item let $\mathcal{D}'_i$ be an $H$-decomposition of $K_{V_i \cup I}^{(3)} \setminus K_{I}^{(3)}$, which exists for $H_{4,2}$ by \cite{bryant}, and for $H_{4,3}$ by Example \ref{eg:H_43-k14-k6}.
\end{itemize}

Let $\mathcal{D}_\infty$ be an $H$-decomposition of $K_{V_n \cup I}^{(3)}$,
  which exists for $H_{4,2}$ by \cite{bryant}, and for $H_{4,3}$ is given by
  the union of $\mathcal{D}'_n$ and a $H_{4,3}$-decomposition of $K_{I}^{(3)}$ (which exists by example \ref{eg:H_43-6}).


For each $1 \leq i < j \leq n$:
\begin{itemize}
  \item let $\mathcal{D}_{i,j}$ be an $H$-decomposition of $L_{V_i,V_j,[\{\infty_1\}]}^{(3)}$, which exists for $H_{4,2}$ by \cite{bryant}, and for $H_{4,3}$ by Example \ref{eg:H_43-l88-k881}; and
  \item let $\mathcal{D}'_{i,j}$ be an $H$-decomposition of $L_{V_i,V_j,[I]}^{(3)}$, which exists for $H_{4,2}$ by \cite{bryant}, and for $H_{4,3}$ by Example \ref{eg:H_43-l88-k886}.
\end{itemize}

It was shown in \cite{hanani} that there exists an $O$-decomposition of $K_{8,8,8}^{(3)}$, so there exists an $H$-decomposition of $K_{8,8,8}^{(3)}$.
Then, for each $1 \leq i < j < k \leq n$, let $\mathcal{D}_{i,j,k}$ be an $H$-decomposition of $K_{V_i,V_j,V_k}^{(3)}$.

Then, if $v \equiv 1$ (mod 8),
\[
    \left( \bigcup_{1 \leq i \leq n} \mathcal{D}_i \right)
    \cup \left( \bigcup_{1 \leq i < j \leq n} \mathcal{D}_{i,j} \right)
    \cup \left( \bigcup_{1 \leq i < j < k \leq n} \mathcal{D}_{i,j,k} \right)
\]
is an $H$-design of order $v = 8n+1$, and if $v \equiv 6$ (mod 8),
\[
    \mathcal{D}_\infty \cup \left( \bigcup_{1 \leq i \leq n-1}
    \mathcal{D}'_i \right) \cup \left( \bigcup_{1 \leq i < j \leq n}
    \mathcal{D}'_{i,j} \right) \cup \left( \bigcup_{1
    \leq i < j < k \leq n} \mathcal{D}_{i,j,k} \right)
\]
is an $H$-design of order $v = 8n+6$. This completes the proof.

\end{proof}

This concludes the proof of Theorem \ref{thm:regular-4}. \qed

Even though there is no $H_{4,2}$-design of order $6$, we can still consider the existence of $H_{4,2}$-designs of order $6$ with index $\lambda \geq 2$.
We have the following reults:

\begin{theorem} \label{thm:H_42-6-simple}
There exists a simple $(6, H_{4,2}, \lambda)$-design if and only if $\lambda \in \{2, 3, 4, 6\}$.
\end{theorem}

\begin{proof}
Let $V = \{0,1,2,3,4,\infty\}$, and let $G$ be the cyclic permutation group $G = \langle (0 \, 1 \, 2 \, 3 \, 4)(\infty) \rangle$.
Define six $H_{4,2}$-blocks on $V$ by:
\begin{align*}
  B_1 &= [\infty, 0, 1, 3, 2, 4]_{H_{4,2}}, \quad
  B_2 = [\infty, 0, 1, 3, 4, 2]_{H_{4,2}}, \quad
  B_3 = [\infty, 0, 1, 4, 2, 3]_{H_{4,2}}, \\
  B_4 &= [\infty, 0, 1, 4, 3, 2]_{H_{4,2}}, \quad
  B_5 =  [\infty, 0, 2, 4, 1, 3]_{H_{4,2}}, \quad
  B_6 = [\infty, 0, 2, 4, 3, 1]_{H_{4,2}}.
\end{align*}
Then, let
\begin{align*}
  \mathcal{D}_2
  &= \mathcal{O}_G(\{B_1, B_2\}) \\
  \mathcal{D}_3
  &= \mathcal{O}_G(\{B_1, B_4, B_5\}) \\
  \mathcal{D}_4
  &= \mathcal{O}_G(\{B_1, B_2, B_3, B_5\}) \\
  \mathcal{D}_6
  &= \mathcal{O}_G(\{B_1, B_2, B_3, B_4, B_5, B_6\}),
\end{align*}
where $\mathcal{O}_G(X)$ denotes the orbit of the set $X$ under $G$.
Then, $\mathcal{D}_\lambda$ is a simple $(6, H_{4,2}, \lambda)$-design for each $\lambda \in \{2, 3, 4, 6\}$.
It remains to show $\lambda \in \{2, 3, 4, 6\}$ is necessary.

If we can show that $\mathcal{D}_6$ is complete, then there cannot exist a simple $(6, H_{4,2}, \lambda)$-design for $\lambda > 6$.
The following three permutations are automorphims of $[1, 2, 3, 4, 5, 6]_{H_{4,2}}$:
\[
  (1 \, 2 \, 3) (4 \, 5 \, 6), \quad (1 \, 2) (4 \, 5), \quad (1 \, 4) (2 \, 5).
\]
The group generated by these three elements has order 24, so $|\Aut(H_{4,2})| \geq 24$.
But, $\mathcal{D}_6$ contains $30 = {6 \choose 6} \frac{6!}{24}$ distinct $H_{4,2}$-blocks,
  so we conclude that $|\Aut(H_{4,2})| = 24$, and $\mathcal{D}_6$ is complete.\footnote{Alternatively,
  one can enumerate all $H_{4,2}$-blocks on 6 vertices (by computer search or other means), and check that there are precisely 30.}

If there existed a simple $(6, H_{4,2}, 5)$-design $\mathcal{D}_5$, then its complement
  $\mathcal{D}_6 \setminus \mathcal{D}_5$ would be a $(6, H_{4,2}, 1)$-design, which does not exist by Theorem \ref{thm:H_42,H_43}.
This completes the proof.
\end{proof}

\begin{theorem} \label{thm:H_42-6}
There exists a (not necessarily simple) $(6, H_{4,2}, \lambda)$-design if and only if $\lambda \geq 2$.
\end{theorem}

\begin{proof}
A $(6, H_{4,2}, 1)$-design does not exist by Theorem \ref{thm:H_42,H_43}, so $\lambda \geq 2$ is necessary.

By Theorem \ref{thm:H_42-6-simple}, there exists a $(6, H_{4,2}, 2)$-design $\mathcal{D}_2$ and a $(6, H_{4,2}, 3)$-design $\mathcal{D}_3$.

For any $\lambda \geq 4$, let $x, y \in \Z_{\geq 0}$ be a solution to $\lambda = 2x + 3y$
  (it is straightforward to see why such a solution must exist).
Then, the collection containing $x$ copies of $\mathcal{D}_2$ and $y$ copies of $\mathcal{D}_3$ is a $(6, H_{4,2}, \lambda)$-design.
\end{proof}




The case $H \simeq H_{5,2}$ remains an open problem, but is discussed in \cite{mesz-rosa}.
The necessary conditions require that $v \equiv 1, 2, 5, 7, 10$ or $11$ (mod $15$).
However, we do have the following results:

\begin{theorem}[\cite{mesz-rosa}] \label{thm:H_52}
An $H_{5,2}$-design of order $v$ exists for all admissible $v \leq 17$, and for
all $v = 4^m + 1$ for $m$ a positive integer.
\end{theorem}

\begin{proof}
There is an $H_{5,2}$-design of order $5$ on the vertex set $\{0,1,2,3,4\}$
  given by $\left\{[0,1,2,3,4]_{H_{5,2}}, [4, 0, 1, 3, 2]_{H_{5,2}} \right\}$.

Hence, if there exists an $S(3, 5, v)$, then there exists an $H_{5,2}$-design of order $v$, given by replacing each block with an $H_{5,2}$-decomposition of $K_5^{(3)}$.
It is known that there exists an $S(3, 5, v)$ for all $v = 4^m+1$, given by spherical geometries (see \cite{khos-laue}), so there is also an $H_{5,2}$-design of order $4^m+1$.

Examples of $H_{5,2}$-designs of order $7$, $10$, and $11$ are given in \cite{mesz-rosa}.
\end{proof}
